%%%%%%%%%%%%%%%%%%%%%%%%%%%%%%%%%%%%%%%%%%%%%%%%%%%%%%%%%%%%%%%%%%%%%%%%%%%%%%%%%%%
%%																			     %%
%%  This template is created by Philipp Ludewig	and its based on the       	     %%
%%  MES template by Jan Boelmann and Shashi Kiran ShivarKumar Honnali			 %%
%%  for further informations please contact: [Mailadress]      			         %%
%%																			     %%
%%																			   	 %%
%%																			     %%
%% 																			     %%
%%                                                                		         %%
%%%%%%%%%%%%%%%%%%%%%%%%%%%%%%%%%%%%%%%%%%%%%%%%%%%%%%%%%%%%%%%%%%%%%%%%%%%%%%%%%%%

%%%%%%%%%%%%%%%%%%%%%%%%%%%%%%%%%%%%%%%%%%%%%%%%%%%%%%%%%%%%%%%%%%%%%%%%%%%%%%
%%%%   						     Preamble                                 %%%%
%%%%%%%%%%%%%%%%%%%%%%%%%%%%%%%%%%%%%%%%%%%%%%%%%%%%%%%%%%%%%%%%%%%%%%%%%%%%%%       
%% here the document will be defined with the settings from above
\documentclass[
    13pt,                % Schriftgroesse 12pt
   a4paper,             % Layout fuer Din A4
   oneside,             % Layout fuer einseitigen Druck
   headinclude,         % Kopfzeile wird Seiten-Layouts mit beruecksichtigt
   headsepline,         % horizontale Linie unter Kolumnentitel
%  footinclude,         % Kopfzeile wird Seiten-Layouts mit beruecksichtigt
%  footsepline,         % horizontale Linie unter Kolumnentitel
%  plainheadsepline,    % horizontale Linie auch beim plain-Style
   plainfootsepline,    % horizontale Linie auch beim plain-Style
   BCOR12mm,            % Korrektur fuer die Bindung
   DIV17,               % DIV-Wert fuer die Erstellung des Satzspiegels, siehe scrguide
%  halfparskip,         % Absatzabstand statt Absatzeinzug
   openany,             % Kapitel können auf geraden und ungeraden Seiten beginnen
   bibliography=totoc,            % Literaturverz. wird ins Inhaltsverzeichnis eingetragen
   listof=totoc,				% Abbildungs und TabellenVZ ins InhaltsVZ
%  pointlessnumbers,    % Kapitelnummern immer ohne Punkt
   numbers=noenddot,
   %tablecaptionabove,   % korrekte Abstaende bei TabellenUEBERschriften
   fleqn,               % fleqn: Glgen links (statt mittig)
   %draft               % Keine Bilder in der Anzeige, overfull hboxes werden angezeigt
   xcolor=table
]{scrbook}

%%%%%%%%%%%%%%%%%%%%%%%%%%%%%%%%%%%%%%%%%%%%%%%%%%%%%%%%%%%%%%%%%%%%%%%%%%%%%%%%%
%%%%% 							Settings        							%%%%%
%%%%%%%%%%%%%%%%%%%%%%%%%%%%%%%%%%%%%%%%%%%%%%%%%%%%%%%%%%%%%%%%%%%%%%%%%%%%%%%%%

%% This will load the parameter of the template like font, size etc.
%% the command "\input{}" loads a file without showing it in the document.
% Alle Pakete um dieses Latex Dokument zu erweitern
% \usepackage{lastpage} 					% Required to determine the last page for the footer
% \usepackage{extramarks} 	 			% Required for headers and footers
\usepackage{graphicx} 					% Required to insert images
\usepackage{listings}					% Required for insertion of code
\usepackage{courier} 					% Required for the courier font
\usepackage[ngerman]{babel}
\usepackage{bibgerm}
\usepackage{wrapfig}
\usepackage{float}
\usepackage{microtype}
\usepackage[utf8]{inputenc}
\usepackage[usenames,dvipsnames]{xcolor}
\usepackage{booktabs}
\usepackage[T1]{fontenc}
\usepackage[square,sort,comma,numbers]{natbib}
\usepackage{lmodern}
% \usepackage{titlesec}
% \usepackage{mdframed}
\usepackage{eurosym}
% \usepackage{calc}
% \usepackage{colortbl}
%\usepackage[scaled]{uarial}
\usepackage[onehalfspacing]{setspace}
\usepackage{acronym}
\usepackage[font=footnotesize]{caption}
\usepackage{url}
\usepackage{breakurl}
\usepackage{hyperref}
\usepackage[automark]{scrpage2}
\usepackage{lipsum}
\usepackage{textcomp}
\usepackage{xcolor,colortbl}
\usepackage{pifont} 			%% For additional special characters available by \verb#\ding{}#
\usepackage{mathtools}


% Sämtliche Metadaten
\newcommand{\sArtofExam}{Leistungsnachweis } % oder Bachelor-Thesis oder Master-Thesis oder Protokoll etc.
\newcommand{\sName}{Hans im Glück}
\newcommand{\sMtrNr}{98654}
\newcommand{\sFirstTutor}{Prof. Dr. Einstein}
\newcommand{\sSecondTutor}{Prof. Dr. Düsentrieb}
\newcommand{\sTitle}{Einfacher Leistungsnachweise schreiben mit LaTeX}
\newcommand{\sCaption}{Latex-Vorlage für Anfänger und Fortgeschrittene}
\newcommand{\TitleDescription}{Latex Vorlage für Bachelor- und Master}
\newcommand{\sModul}{Modules}
\newcommand{\sCompany}{DuckTales AG}
\newcommand{\sDegree}{Bachelor}
\newcommand{\sFaculty}{Wirtschaftsinformatik / Informatik}
\newcommand{\sDepartment}{Fachbereich 2}
\newcommand{\sUni}{Hochschule Bremerhaven}
\newcommand{\sLocation}{Bremerhaven}
\newcommand{\sTime}{April 2017}
\newcommand{\sVersion}{version 0.1}
\newcommand{\sSubmissiondate}{2. April 2017}




% Einstellungen zu den Farben
\input{lib/colors-and-codestyle}

% Einstellungen zu den Seiten
\pagestyle{fancy} %eigener Seitenstil
\fancyhf{} %alle Kopf- und Fußzeilenfelder bereinigen
\fancyhead[L]{} %Kopfzeile links
\fancyhead[C]{} %zentrierte Kopfzeile
\fancyhead[R]{} %Kopfzeile rechts
\renewcommand{\headrulewidth}{1.0pt} %obere Trennlinie
\fancyfoot[C]{\thepage} %Seitennummer
\renewcommand{\footrulewidth}{1.0pt} %untere Trennlinie

%\pagestyle{scrheadings}
%\ihead[]{\headmark}
%\chead[]{}
%\ohead[]{}
%\ifoot[]{}
%\cfoot[]{}
%\ofoot[\pagemark]{\pagemark}
%\setheadsepline{1.0pt}
%\setfootsepline{1.0pt}

\hypersetup{
	breaklinks = {true}, %Erlaubt Zeilenumbrüche in Links
	colorlinks = {true}, %Benutze farbige Links
	citecolor = {DocumentLinkColor}, %Farbe für Zitate
	linkcolor = {DocumentLinkColor}, %beeinflusst Inhaltsverzeichnis und Seitenzahlen
	urlcolor = {DocumentLinkColor}, %Weblink-Farbe
	pdftitle = {Konzeption und Entwicklung eines hochverfügbaren VPN-Clusters auf Basis von Debian Linux}, %Titel der Arbeit
	pdfsubject = {Bachelor-Thesis}, %Thema der Arbeit
	pdfauthor = {Jan Torben Bein}, %AutorIN der Arbeit
	pdfkeywords = {Cluster HA Linux Debian OpenVPN VPN Corosync Pacemaker Heartbeat}, %Stichwörter zur Arbeit
	pdfproducer = {pdfLaTeX}, %Erzeugt durch
	pdfcreator = {MiKTeX}, %Erstellt mit
	pdfstartview = {FitV},
	pdfview = {FitH},
	pdffitwindow = {true}
}
\def\UrlBreaks{\do\/\do-}

\setlength\parindent{0pt} % Removes all indentation from paragraphs

% Content and Stuff
\setcounter{tocdepth}{3}
\setcounter{secnumdepth}{3}

% Fonts and Style
\renewcommand*\familydefault{\sfdefault} %% Only if the base font of the document is to be sans serif

\addto\captionsngerman{
	\renewcommand{\figurename}{Abb.}
	\renewcommand{\tablename}{Tab.}
}

\setlength{\footskip}{2\baselineskip}

\fussy


%Messbox zur Druckkontrolle:
\newcommand{\Messbox}[2]{% Parameters: #1=Breite, #2=Hoehe
	\setlength{\unitlength}{1.0mm}%
	\begin{picture}(#1,#2)%
	\linethickness{0.05mm}%
	\put(0,0){\dashbox{0.2}(#1,#2)%
		{\parbox{#1mm}{%
				\centering\footnotesize 
				%{\bf MESSBOX}\\ 
				Breite $ = #1 {\rm\ mm}$\\
				H\"ohe $ = #2 {\rm\ mm}$
	}}}\end{picture}
}





%%%%%%%%%%%%%%%%%%%%%%%%%%%%%%%%%%%%%%%%%%%%%%%%%%%%%%%%%%%%%%%%%%%%%%%%%%%%%%%%%
%%%%% 						Documents Section       			     		%%%%%
%%%%%%%%%%%%%%%%%%%%%%%%%%%%%%%%%%%%%%%%%%%%%%%%%%%%%%%%%%%%%%%%%%%%%%%%%%%%%%%%%
\begin{document}
%----------------------------------------------------------------------------------------
%	TITLE-Page
%----------------------------------------------------------------------------------------
	\begin{titlepage}	
	\begin{figure}[h]
		\flushright
    	\includegraphics[scale=0.4]{images/hs-logo.png}
	\end{figure}
	\vspace*{1.0in}
	\hspace*{-1.3in}\colorbox{TitleBox}{
		\centering
		\parbox[t]{1.0\paperwidth}{
			\vspace*{0.7in}
			
			\centering
			\Large\textbf{\sTitle} \\

			\vspace*{0.7cm}
			\large{\sCaption}
			
			\vspace*{0.7in}
			
		}
	}
	\vfill{\centering \large
		\hfill \sName \\
		\hfill \sMtrNr \\
		\hfill \sUni \\
		\hfill \sDepartment \hspace{2mm}- \sFaculty \\
	}
\end{titlepage}

\newpage

\thispagestyle{empty}
\vspace*{2.0in}
\parbox[t]{1.0\linewidth}{ 
	\vspace*{0.7cm}
	
	\centering
	\Large\textbf{\sTitle} \\
	\vspace*{0.7cm}
	\large{\TitleDescription}
	

}
\vfill{
	\sArtofExam eingereicht im Rahmen des \sModul \\
	im Studiengang \sFaculty \hfill \\
	im \sDepartment\hfill \\
	der \sUni \hfill \\
	\\
	Betreuender Prüfer: \sFirstTutor \hfill \\
	Zweitgutachter: \sSecondTutor \hfill \\
	\\
	Abgabe am \sSubmissiondate

	\begin{titlepage} %% start document
	\centering %% bring everything in center of dokument
    
    \includegraphics[scale = 0.4]{images/hs-logo.png}\\[1.5 cm]	%% quick way to include a graphic. DO NOT USE FOR FIGURES, SEE DOCUMENTATION HOW TO USE FIGURES 
    % Text in [] will give space between the lines
    \textsc{\LARGE University of Applied Sciences}\\[2.0 cm]%% University Name
	\textsc{\Large \sFaculty}\\[0.5 cm]				%% Course name
	\textsc{\large \sFirstTutor}\\[0.5 cm] %% Name of professor
    	
	\rule{\linewidth}{0.2 mm} \\[0.4 cm] %% create a line
	{ \huge \bfseries {\sTitle}}\\
	\rule{\linewidth}{0.2 mm} \\[1.5 cm] %% create a line
    %change here the distance when more than four authors
	
	\begin{minipage}{0.5\textwidth} %% create a table for authors and mat. numbers
		\begin{center} \large %% you can remove "\large" if many authors for more space
			\emph{Author:}\\
			\sName
			\end{center}
			\end{minipage}~
			\begin{minipage}{0.5\textwidth}
			\begin{center} \large %% you can remove "\large" if many authors for more space
			\emph{Mat. Number:} \\
            \sMtrNr								
		\end{center}
	\end{minipage}\\[1.5 cm] %% reduce here space between date to have it on titlepage when more than four authors
	
	{ \large Bremerhaven, the \today} %% show where and when (date)
    
%% end of document
\end{titlepage}
	\newpage

%----------------------------------------------------------------------------------------
%	ABSTRACT
%----------------------------------------------------------------------------------------
	\addchap*{Abstract}
\thispagestyle{empty}

 \addsec*{Thema}
 \sTitle

 \addsec*{Stichworte}
 Hier Stichworte, Stichworte, Stichworte, Stichworte

\addsec*{Kurzzusammenfassung}
In diesem Dokument findet ihr eine Erklärung und Vorlage zu LaTeX. Da auf English: \url{https://github.com/VoLuong/Begin-Latex-in-minutes}.

	\newpage
	
%----------------------------------------------------------------------------------------
%	Explanation
%----------------------------------------------------------------------------------------
	\addchap*{Erklärung}
\thispagestyle{empty}

Hiermit erkläre ich gegenüber dem \sDepartment der \sUni,

\begin{itemize}
	\item dass die vorliegende Arbeit mit dem Thema "\sTitle : \sCaption" \ von mir
	persönlich, selbstständig und ausschließlich unter Zuhilfenahme der im Literaturverzeichnis genannten Werke und
	Dokumente angefertigt wurde und dass keine fremde Hilfe in Anspruch genommen haben.
	\item dass die Arbeit weder vollständig noch in Teilen von mir selbst noch von anderen als Leistungsnachweis andernorts eingereicht wurde.
	\item dass ich wörtlich oder sinngemäß übernommene Textteile aus Schriften anderer Autoren als Zitate
	gekennzeichnet und die jeweilige Quelle im Literaturverzeichnis am Ende der Arbeit aufgeführt habe.
	\item dass ich alle Zeichnungen, Skizzen, Grafiken, Illustrationen, Fotografien und sonstige bildlichen	Darstellungen jeder Art sowie Ton - und Datenträger anderer Urheber als Übernahmen gekennzeichnet und die jeweilige Quelle im Literaturverzeichnis am Ende der Arbeit aufgeführt habe.
\end{itemize}

Mir ist bekannt, dass gegebenenfalls eine Überprüfung der hier vorgelegten Arbeit mittels einer
Antiplagiat-Software vorgenommen wird. Dafür stelle ich auf Nachfrage eine digitale, durchsuchbare Kopie dieser Arbeit zur Verfügung.

Mir ist bekannt, dass die Einreichung einer Arbeit unter Verwendung von Material, welches nicht als das geistige Eigentum anderer Personen gekennzeichnet wurde, ernsthafte Konsequenzen nach sich zieht.

\vspace {0.5cm}

Bremerhaven, \today \qquad

\begin{minipage}[hbt]{3in}
\vspace {0.5cm}
	\includegraphics[scale=0.15]{images/signatur.png} \\
	\sName \ (\sMtrNr)
\end{minipage}
	\newpage
	
%----------------------------------------------------------------------------------------
%	TABLE OF CONTENTS
%----------------------------------------------------------------------------------------
	\pagenumbering{Roman} 
	\setcounter{page}{1}
	\begin{spacing}{1.0}
 		\tableofcontents
 		\newpage 
 		
	\end{spacing}
	\newpage

%----------------------------------------------------------------------------------------
%	GLOSSAR
%----------------------------------------------------------------------------------------
	\addchap{Glossar}

\begin{acronym}
	\acro{AD}{Active Directory}\\
	Name des Verzeichnisdienstes von Microsoft Windows Server. Ab der Version 2008 umbenannt in \textbf{Active Directory Domain Services} (ADDS)
	\acro{AES}{Advanced Encryption Standard}
	\acro{AIS}{Application Interface Specification}
	\acro{ARP}{Address Resolution Protocol}\\
	Ein Netzwerkprotokoll welches zu einer IP-Adresse die physikalische Adresse zuordnet und ggf. in einer Tabelle hinterlegt.
\end{acronym}
	\newpage

%----------------------------------------------------------------------------------------
%	CHAPTERs
%----------------------------------------------------------------------------------------
	\setcounter{page}{1}
	\pagenumbering{arabic}
	\chapter{Einleitung}
Willkommen zur Einführung in die Vorlage für eure schriftlichen Leistungsnachweise bzw. Bachelor- oder Masterbschlussarbeiten. Wir werden gemeinsam den Inhalt der Latex Vorlage durchgehen und dabei die Vorteile und Möglichkeiten von Latex ergründen. Beginnen wir mit der Dateistruktur dieser Vorlage.

\begin{itemize}
	\item Vorlage/
		\begin{itemize}
			\item bib/ - Unter diesem Ordner findet ihr die Bibtex Datei welche ihr mit Programmen wie readcube oder Jabref bearbeiten könnt. \url{http://www.jabref.org/}. Außerdem interessant für alle \url{https://www.mendeley.com/} und \url{http://www.citeulike.org/}
			\item chapter/ - In diesem Ordner werden sämtliche Latex .tex Dateien abgelegt in denen ihr euren Text verfasst. Ein Tipp: verteilt die Kapitel und Sektionen auf verschiedene Dateien. Dies ermöglicht ein paralleles Arbeiten und erleichtert die Arbeit an längeren Texten.
			\item images/ - Wie der Name schon sagt, ist dies der Ordner für Bilder aller Art. Benutzt dabei am besten .png, .svg oder .pdf da diese Dateiformate verlustfrei beim skalieren der Größe sind.
			\item lib/ - Hier sind sämtliche Dateien abgelegt mit denen das Aussehen des Dokumentes verändert werden kann. 
		\end{itemize}
\end{itemize}

Wenn ihr \textbf{komplette Neulinge} in LaTeX seid schaut gleichzeitig hier vorbei: \\\url{http://latex.tugraz.at/latex/tutorial}\\
\url{https://github.com/VoLuong/Begin-Latex-in-minutes}\\
http://texwelt.de/wissen/\\
Für alle die im Browser kollaborativ arbeiten möchten empfiehlt es sich \url{www.overleaf.com} zu nutzen. Wenn man lieber auf seinem PC arbeiten möchte, ist unter Windows die Tex-Umgebung \url{miktex.org} zu installieren. Auf dem Betriebssystem Linux muss in einem Terminal folgender Befehl eingegeben werden: "'sudo apt-get install texlive-full"'. Damit werden sämtliche Pakete und Sprachen heruntergeladen. Auf beiden Betriebssystemen bietet es sich an den Tex-Editor \emph{TexStudio} zu nutzen. Es gibt jedoch noch andere: \url{https://en.wikipedia.org/wiki/Comparison_of_TeX_editors}.

Bei weiteren Fragen oder Problemem zur Installation von Tex hier ein Link:\\
\url{http://texwelt.de/wissen/fragen/11038/wie-installiere-ich-latex}

\section{Wie beginnt man?}

Ein jedes Dokument beginnt man damit, die Metadaten zu bearbeiten und das Titelblatt einzustellen. In der Index TeX Datei "'Index.tex"' findet man dazu unter dem Abschnitt TITLE-Page zwei \textbf{input} Befehle die zu den Seiten \textbf{titlepage\_alone} und \textbf{titlepage\_group} verweisen. Wie die Namen schon aussagen ist jeder der Seiten für eine Sache optimiert. Bei einer Gruppe ab drei Personen wird empfohlen die Gruppenversion zu nehmen. Die Auswahl ist natürlich jedem selbst überlassen.\\

Weiter geht es zu den Metadaten in der Tex-Datei \textbf{person-configuration} in dem Ordner \emph{lib/}. Dort sollten sämtliche Informationen hinter den Befehlen angepasst werden. Die ersten Veränderungen können dann bereits auf dem Deckblatt verzeichnet werden.


\section{Aufbau der Index Datei}

Die Index-Datei beinhaltet die Struktur des Dokumentes. Sie ist eingeteilt in die Abschnitte:
\begin{itemize}
	\item Preamble: Hier werden die Einstellungen für das 'scrbook' übergeben. Wenn ihr eine anderes Aussehen möchtet, probiert stattdessen 'article'
	\item Einstellungen: Hier werden die Einstellungen zu Packages, Nutzerinformationen und Aussehen der Seiten geladen
	\item Bereich des Dokumentes: In diesem Segment wird die Abfolge des Dokumentes festgelegt. 
	\begin{itemize}
		\item Title-Page: Es stehen zwei Arten von Deckblättern zur Verfügung. Für Einzel- und Gruppenarbeiten.
		\item ABSTRACT: Auf dieser Seite muss das Thema des Leistungsnachweises kurz und bündig beschrieben werden.
		\item Explanation: In der Erklärung weist man darauf hin, dass kein Plagiat angefertigt wurde. Vergesst nicht zu unterschreiben bevor ihr abgebt.
		\item TABLE OF CONTENTS: Unter diesem Abschnitt wird festgelegt in welcher Reihenfolge die Gliederungen geordnet werden. Dabei ist interessant, dass mit dem Befehl \emph{pagenumbering} die Nummerierung auf "'Roman"' gesetzt wird und mit \emph{setcounter} die Zählung bei 1 zurückgesetzt wird.
		\item GLOSSAR: Hier sollten Fremdbegriffe welche im Dokument vorkommen mit kurzer Definition erklärt werden, sodass die Abkürzungen im Dokument benutzt werden kann.
		\item CHAPTERS: Hier bindet ihr eure Kapitel und Abschnitte ein.
		\item BIBLIOGRAPHY: Literaturverzeichnis wird am besten mit Jabref gepflegt
		\item APPENDIX: Wenn ihr einen Anhang benötigt für größere Bilder oder Diagramme, ist hier der richtige Ort dafür.
	\end{itemize}
\end{itemize}


\section{Pakete und CTAN}

Im Umgang mit LaTex werden sie sehr schnell auf das Problem stoßen das sie etwas tun möchten was von den in diesem Template vorhandenen Paketen nicht unterstützt wird. Um die Funktionalität hinzuzufügen suchen sie über eine Suchplattform (DuckDuckGo, Google etc.) ihrer Wahl nach dem richtigen \LaTeX Paket. In der Datei "'/lib/packages.tex"' sollten sie das Paket mittels des Befehls \verb|\usepackage{package}| einbinden. Sie können diesen Befehl überall im Dokument platzieren, der Übersicht halber sollte dies jedoch in der "'/lib/packages.tex"' geschehen. Nachdem das Paket eingebunden ist werden sie beim nächsten Compilieren ihres Dokumentes gefragt ob das Paket heruntergeladen werden soll. Bei der Installation haben sie bereits ein Repository einer unabhängigen Universität oder Hochschule ausgewählt. Jedes Paket hat ein Dokumentation, welche auf der Webseite \url{https://ctan.org} zu finden ist. Es empfiehlt sich stets die Dokumentation zu lesen, da diese meistens jede aufkommende Frage beantwortet. 
	\chapter{Drucken der Arbeit}
\label{chap:Drucken}



\section{Drucken}

\subsection{Drucker und Papier}

Die Arbeit sollte in der Endfassung unbedingt auf einem
qualitativ hochwertigen Laserdrucker ausgedruckt werden, Ausdrucke
mit Tintenstahldruckern sind \emph{nicht} ausreichend. Auch das
verwendete Papier sollte von guter Qualität (holzfrei) und
üblicher Stärke (mind.\ $80\; {\mathrm g} / {\mathrm m}^2$) sein.
Falls \emph{farbige} Seiten notwendig sind, sollte man diese einzeln%
\footnote{Tip: Mit \emph{Adobe Acrobat} lassen sich sehr einfach einzelne Seiten
des Dokuments für den Farbdruck auswählen und zusammenstellen.}
auf einem Farb-Laserdrucker ausdrucken und dem Dokument beifügen.

Übrigens sollten \emph{alle} abzugebenden Exemplare {\bf
gedruckt} (und nicht kopiert) werden! Die Kosten für den Druck
sind heute nicht höher als die für Kopien, der
Qualitätsunterschied ist jedoch --  bei Bildern und Grafiken
-- meist deutlich.


\subsection{Druckgröße}

Zunächst sollte man sicherstellen, dass die in der fertigen PDF-Datei eingestellte
Papiergröße tatsächlich \textbf{A4} ist! Das geht z.B.\ mit \emph{Adobe Acrobat}
oder \emph{SumatraPDF}
über \texttt{File} $\rightarrow$ \texttt{Properties},
wo die Papiergröße des Dokuments angezeigt wird:
\begin{center}
\textbf{Richtig:} A4 = $8{,}27 \times 11{,}69$ in bzw.\ $21{,}0 \times 29{,}7$ cm.
\end{center}

Ein häufiger und leicht zu übersehender Fehler beim Ausdrucken von
PDF-Dokumenten wird durch die versehentliche Einstellung der
Option "`Fit to page"' im Druckmenü verursacht, wobei die Seiten
meist zu klein ausgedruckt werden. Überprüfen Sie daher die Größe
des Ausdrucks anhand der eingestellten Zeilenlänge oder mithilfe
einer Messgrafik, wie am Ende dieses Dokuments gezeigt.
Sicherheitshalber sollte man diese Messgrafik bis zur
Fertigstellung der Arbeit beizubehalten und die entsprechende
Seite erst ganz am Schluss zu entfernen.
Wenn, wie häufig der Fall, einzelne Seiten getrennt in Farbe gedruckt 
werden, so sollten natürlich auch diese genau auf die Einhaltung der Druckgröße 
kontrolliert werden!




\section{Binden}

Die Endfassung der Leistungsnachweises ist je nach Vorgaben des Studiengangs, meist eine einfache Bindung einzureichen. Diese kann man im ASTA-Büro oder in Copyshops drucken bzw. binden lassen.

Falls man im Falle einer Bachelorarbeit oder Masterarbeit die Arbeit bei einem
professionellen Buchbinder durchführen lässt, sollte man auch auf
die Prägung am Buchrücken achten, die kaum zusätzliche Kosten
verursacht. Üblich ist dabei die Angabe des Familiennamens des
Autors und des Titels der Arbeit. Ist der Titel der Arbeit zu
lang, muss man notfalls eine gekürzte  Version angeben, wie z.B.:
%
\begin{center}
\setlength{\fboxsep}{3mm}
\fbox{
\textsc{Schlaumeier}
\textperiodcentered\ \textsc{Parz.\ Lösungen zur allg.\ Problematik}}
\end{center}
%



\section{Elektronische Datenträger (CD-R, DVD)}
Speziell bei Arbeiten im Bereich der Informationstechnik (
nicht nur dort) fallen fast immer Informationen an, wie Programme,
Daten, Grafiken, Kopien von Internetseiten usw., die für eine
spätere Verwendung elektronisch verfügbar sein sollten.
Vernünftigerweise wird man diese Daten während der Arbeit bereits
gezielt sammeln und der fertigen Arbeit auf einer CD-ROM oder DVD
beilegen. Es ist außerdem sinnvoll -- schon allein aus Gründen der
elektronischen Archivierbarkeit -- die eigene Arbeit selbst als
PDF-Datei beizulegen.%
\footnote{Auch Bilder und Grafiken könnten in elektronischer Form nützlich
sein, die LaTeX- oder Word-Dateien sind hingegen überflüssig.}


Falls ein elektronischer Datenträger (CD-ROM oder DVD) beigelegt
wird, sollte man auf folgende Dinge achten:
%
\begin{enumerate}
\item Jedem abzugebenden Exemplar muss eine identische Kopie des
Datenträgers beiliegen. %
\item Verwenden Sie qualitativ hochwertige Rohlinge und überprüfen
Sie nach der Fertigstellung die tatsächlich gespeicherten Inhalte
des Datenträgers! %
\item Der Datenträger sollte in eine im hinteren Umschlag
eingeklebte Hülle eingefügt sein und sollte so zu entnehmen sein,
dass die Hülle dabei \emph{nicht} zerstört wird (die
meisten Buchbinder haben geeignete Hüllen parat). %
\item Der Datenträger muss so beschriftet sein, dass er der
Leistungsnachweis eindeutig zuzuordnen ist, am Besten durch ein
gedrucktes Label%
\footnote{Nicht beim lose abgegebenen Bibliotheksexemplar --
dieses erhält ein standardisiertes Label durch die Bibliothek.} %
oder sonst durch \emph{saubere}
Beschriftung mit
der Hand und einem feinen, wasserfesten Stift. %
\item Nützlich ist auch ein (grobes) Verzeichnis der Inhalte des
Datenträgers
\end{enumerate}

	\chapter[Mathem.\ Formeln etc.]{Mathematische Formeln, Gleichungen und Algorithmen}
\label{chap:Mathematik}

% Formelverzeichniss

Das Formatieren von mathematischen Elementen gehört sicher zu den
Stär\-ken von LaTeX. Man unterscheidet zwischen mathematischen Elementen
im Fließtext und freistehenden Gleichungen, die in der Regel
fortlaufend nummeriert werden. Analog zu Abbildungen und Tabellen sind dadurch
Querverweise zu Gleichungen leicht zu realisieren. Zum besseren Verständnis schaut euch neben dieser PDF auch den Quelltext in der Datei \emph{mathematik.tex} an.


\section{Mathematische Elemente im Fließtext}

Mathematische Symbole, Ausdrücke, Gleichungen etc.\ werden im Fließtext durch paarweise \verb!$! \ldots \verb!$! markiert. Hier ein Beispiel:
%
\begin{quote}
Der Nah-Unendlichkeitspunkt liegt bei
$\bar{a} = f' \cdot \bigl( \frac{f'}{K \cdot u_{\max}} + 1 \bigr)$,
sodass bei einem auf $\infty$ eingestellten Objektiv von der Entfernung
$\bar{a}$ an alles scharf ist. Fokussiert man das
Objektiv auf die Entfernung $\bar{a}$ (dass heißt, $a_0 = \bar{a}$), dann wird
im Bereich $[\frac{\bar{a}}{2}, \infty]$ alles scharf.
\end{quote}
%
Dabei sollte man unbedingt darauf achten, dass die Höhe der einzelnen Elemente im Text nicht zu groß wird. 

\paragraph{Häufiger Fehler:} 
Im Fließtext wird bei einfachen Variablen oft auf die Verwendung der richtigen, mathematischen Zeichen vergessen, wie etwa in 
"`X-Achse"' anstelle von "`$X$-Achse"' (\verb!$X$-Achse!).



\section{Freigestellte Ausdrücke}

Freigestellte mathematische Ausdrücke können in LaTeX im einfachsten Fall durch paarweise \verb!$$! \ldots \verb!$$! erzeugt werden. Das Ergebnis wird zentriert, erhält jedoch keine Numerierung. So ist z.B.\ $$ y = 4 x^2 $$ das Ergebnis von \verb!$$ y = 4 x^2 $$!.

\subsection{Einfache Gleichungen} 

Meistens verwendet man in solchen Fällen jedoch die \texttt{equation}-Umgebung zur Herstellung numerierter Gleichungen, auf die man im Text jederzeit verweisen kann. Zum Beispiel erzeugt
%

\begin{equation}
  f(k) = \frac{1}{N} \sum_{i=0}^{k-1} i^2 . 
  \label{eq:MyFirstEquation}
\end{equation}

%
die Gleichung
%
\begin{equation}
  f(k) = \frac{1}{N} \sum_{i=0}^{k-1} i^2 . 
\label{eq:MyFirstEquation}
\end{equation}
%
Mit \verb!\ref{eq:MyFirstEquation}! erhält man wie üblich die Nummer (\ref{eq:MyFirstEquation}) dieser Gleichung (siehe dazu auch Abschn.\ \ref{sec:VerweiseAufGleichungen}). 
Dieselbe Gleichung \emph{ohne} Numerierung kann man übrigens mit der \texttt{equation*}-Umgebung erzeugen.



\begin{center}
\setlength{\fboxrule}{0.2mm}
\setlength{\fboxsep}{2mm}
\fbox{%
\begin{minipage}{0.9\textwidth}
Man beachte, dass \textbf{Gleichungen} inhaltlich ein \textbf{Teil des Texts} sind und daher neben der sprachliche \textbf{Überleitung} auch die \textbf{Punktuation} (wie in Gl.\ \ref{eq:MyFirstEquation} gezeigt) beachtet werden muss. Bei Unsicherheiten sollte man sich passende Beispiele in einem guten Mathematik\-buch ansehen.
\end{minipage}}
\end{center}
%
Für Interessierte findet sich mehr zum Thema Mathematik und Prosa in \cite{Mermin89} und \cite{Higham98}.

\subsection{Mehrzeilige Gleichungen}

Für mehrzeilige Gleichungen bietet LaTeX die 
\verb!eqnarray!-Umgebung, die allerdings etwas eigenwillige Zwischenräume erzeugt.
Es empfiehlt sich, dafür gleich auf die erweiterten Möglichkeiten des \texttt{amsmath}-Pakets%
\footnote{American Mathematical Society (AMS). \texttt{amsmath} ist Teil der LaTeX Standardinstallation und wird von \texttt{hgb.sty} bereits importiert.}
\cite{amsldoc02} zurückzugreifen.
Hier ein Beispiel mit zwei am $=$ Zeichen ausgerichteten Gleichungen,
%
\begin{align}
f\_1 (x,y) &= \frac{1}{1-x} + y , \label{eq:f1} \\
f\_2 (x,y) &= \frac{1}{1+y} - x , \label{eq:f2}
\end{align}
\formelentry{Gleichung Gesamt}
%
erzeugt mit der \texttt{align}-Umgebung aus dem \texttt{amsmath}-Paket:
%

\begin{align}
  f\_1 (x,y) &= \frac{1}{1-x} + y , \label{eq:f11} \\
  \formelentry{Formel 1}
  f\_2 (x,y) &= \frac{1}{1+y} - x , \label{eq:f21}
  \formelentry{Formel 2}
\end{align}



\subsection{Fallunterscheidungen}

Mit der \texttt{cases}-Umgebung aus \texttt{amsmath} sind Fallunterscheidungen, unter anderem innerhalb von Funktionsdefinitionen, sehr einfach zu bewerkstelligen. Beispielsweise wurde die rekursive Definition
%
\begin{equation}
	f(i) =
	\begin{cases}
	  0             & \text{für $i = 0$},\\
	  f(i-1) + f(i) & \text{für $i > 0$}.
	\end{cases}
\end{equation}
mit folgenden Anweisungen erzeugt:
%

\begin{equation}
	f(i) =
	\begin{cases}
	  0             & \text{für $i = 0$},\\
	  f(i-1) + f(i) & \text{für $i > 0$}.
	\end{cases}
\end{equation}

%
Man beachte dabei die Verwendung des sehr praktischen \verb!\text{..}!-Makros, mit dem im Mathematik-Modus gewöhnlicher Text eingefügt werden kann, sowie wiederum die Punktuation innerhalb der Gleichung.

\subsection{Gleichungen mit Matrizen}

Auch hier bietet \texttt{amsmath} einige Vorteile gegenüber der Verwendung der LaTeX Standardkonstrukte. Dazu ein einfaches Beispiel für die Verwendung der \texttt{pmatrix}-Umgebung für Vektoren und Matrizen,
%
\begin{equation}
	\begin{pmatrix} x' \\ y' \end{pmatrix}
	= 
	\begin{pmatrix}
	  \cos \phi & -\sin \phi \\
	  \sin \phi & \phantom{-}\cos \phi
	\end{pmatrix} 
	\cdot
	\begin{pmatrix}	x \\ y \end{pmatrix} ,
\end{equation}
%
das mit den folgenden Anweisungen erzeugt wurde:
%

\begin{equation}
	\begin{pmatrix} 
			x' \\ 
			y' 
	\end{pmatrix}
	= 
	\begin{pmatrix}
		  \cos \phi & -\sin \phi \\
		  \sin \phi & \phantom{-}\cos \phi /+ \label{lin:phantom} +/
	\end{pmatrix} 
	\cdot
	\begin{pmatrix} 
			x \\ 
			y 
	\end{pmatrix} ,
\end{equation}
\formelentry{Gleichung}
%
Ein nützliches Detail darin ist das TeX-Makro \verb!\phantom{..}! (in Zeile \ref{lin:phantom}), das sein Argument unsichtbar einfügt und hier als Platzhalter für das darüberliegende Minuszeichen verwendet wird. Alternativ zu \texttt{pmatrix} kann man mit der \texttt{bmatrix}-Umgebung Matrizen
und Vektoren mit eckigen Klammern erzeugen.
Zahlreiche weitere mathematische Konstrukte des \texttt{amsmath}-Pakets sind in \cite{amsldoc02} beschrieben.



\subsection{Verweise auf Gleichungen}
\label{sec:VerweiseAufGleichungen}

Beim Verweis auf nummerierte Formeln und Gleichungen genügt grundsätzlich die Angabe 
der entsprechenden Nummer in runden Klammern,
z.B.
\begin{center}
%"`\ldots\ wie aus (\ref{eq:f1}) abgeleitet werden kann \ldots"'
"`\ldots\ wie aus (\ref{eq:f1}) abgeleitet werden kann \ldots"'
\end{center}
Um Missverständnisse zu vermeiden, sollte man in Texten mit
nur wenigen mathematischen Elementen -- "`Gleichung \ref{eq:f1}"', "`Gl.~\ref{eq:f1}"' 
oder "`Gl.~(\ref{eq:f1})"' schreiben (natürlich konsistent). 
%\emph{Falsch} wäre hingegen "`Gleichung (\ref{eqn:zerstreuungskreis})"'.

\begin{center}
\setlength{\fboxrule}{0.2mm}
\setlength{\fboxsep}{2mm}
\fbox{%
\begin{minipage}{0.9\textwidth}
\textbf{Achtung:} Vorwärtsverweise auf (im Text weiter hinten liegende) Gleichungen sind \textbf{äußerst ungewöhnlich} 
und sollten vermieden werden! Glaubt man dennoch so etwas zu benötigen, dann wurde
meistens ein Fehler in der Anordnung gemacht.
\end{minipage}}
\end{center}


\section{Spezielle Symbole}

\subsection{Zahlenmengen}
Einige häufig verwendete Symbole sind leider im ursprünglichen
mathematischen Zeichensatz von LaTeX nicht enthalten, z.B. die
Symbole für die reellen und natürlichen Zahlen. 


\subsection{Operatoren}

In LaTeX\ sind Dutzende von mathematischen Operatoren für spezielle Anwendungen definiert. Am häufigsten werden natürlich die arithmetischen Operatoren $+$, $-$, $\cdot$ und $/$ benötigt. Ein dabei oft beobachteter Fehler (der wohl aus der Programmierpraxis resultiert) ist die Verwendung von $*$ für die einfache Multiplikation -- richtig ist $\cdot$ (\verb!\cdot!).%
\footnote{Das Zeichen $*$ wird üblicherweise für den Faltungsoperator verwendet.}
%
Für Angaben wie z.B. "`ein Feld mit $25 \times 70$ Metern"' (aber auch fast \emph{nur} dafür) verwendet man sinnvollerweise den $\times$ (\verb!\times!) Operator und \emph{nicht} einfach das Textzeichen~"`x"'!


\subsection{Variable (Symbole) mit mehreren Zeichen}
Vor allem bei der mathematischen Spezifikation von Algorithmen und Programmen
ist es häufig notwendig, Symbole (Variablennamen) mit mehr als einem Zeichen
zu verwenden, z.B.
%
$$Scalefactor\leftarrow Scalefactor^2 \cdot 1.5 \; ,$$
%
\textbf{fälschlicherweise} erzeugt durch 
\begin{quote}
	\verb!$Scalefactor \leftarrow Scalefactor^2! \verb!\cdot 1.5$!.
\end{quote}
Dabei interpretiert LaTeX allerdings die Zeichenkette "`Scalefactor"' als 11 einzelne,
aufeinanderfolgende Symbole $S$, $c$, $a$, $l$, $e$, \ldots und setzt dazwischen
entsprechende Abstände.
\textbf{Richtig} ist, diese Buchstaben mit
\verb!\mathit{..}! zu \emph{einem} Symbol zusammenzufassen.
Der Unterschied ist in diesem Fall deutlich sichtbar:
%
\begin{center}
\setlength{\tabcolsep}{4pt}
\begin{tabular}{llll}
\text{Falsch:}   & $Scalefactor^2$ & $\leftarrow$ & \verb!$Scalefactor^2$! \\
\text{Richtig:}  & $\mathit{Scalefactor}^2$ & $\leftarrow$ & \verb!$\mathit{Scalefactor}^2$!
\end{tabular}
\end{center}
%
Grundsätzlich sollte man aber derart lange Symbolnamen aber ohnehin vermeiden und stattdessen 
möglichst kurze (gängige) Symbole verwenden
(z.B.\ Brennweite $f = 50 \, \mathrm{mm}$ statt $\mathit{Brennweite} = 50 \, \mathrm{mm}$).

\subsection{Funktionen}

Während Symbole für Variablen traditionell (und in LaTeX\ automatisch) \emph{italic} gesetzt werden, verwendet man für die Namen von Funktionen und Operatoren üblicherweise
\emph{roman} als Schrifttyp, wie z.B. in
\begin{center}
\begin{tabular}{lcl}
	$\sin \theta = \sin(\theta + 2 \pi)$ & 
	$\leftarrow$ & \verb!$\sin \theta = \sin(\theta + 2 \pi)$! \\
	\end{tabular}
\end{center}
Das ist bei den bereits vordefinierten Standardfunktionen (wie
\verb!\sin!,
\verb!\cos!,
\verb!\tan!,
\verb!\log!,
\verb!\max!
) automatisch der Fall.
Diese Konvention sollte man auch bei selbstdefinierten Funktionen befolgen,
wie etwa in
\begin{center}
	\begin{tabular}{lcl}
	$\mathrm{Distance}(A,B) = |A-B|$ & $\leftarrow$ & \verb!$\mathrm{Distance}(A,B) = |A-B|$! \\
	\end{tabular}
\end{center}


\subsection{Maßeinheiten und Währungen}

Bei der Angabe von Maßeinheiten wird üblicherweise Normalschrift
(keine Italics) verwendet, z.B.:
\begin{quote}
Die Höchstgeschwindigkeit der \textit{Bell XS-1} beträgt 345~m/s
bei einem Startgewicht von 15~t. 
Der Prototyp kostete über 25.000.000 US\$, also ca.\ 19.200.000 \euro\ nach heutiger Umrechnung.
\end{quote}
Der Abstand zwischen der Zahl und der Maßeinheit ist dabei
gewollt.
Das \$-Zeichen erzeugt man mit \verb!\$! und
das Euro-Symbol (\euro) mit dem Makro \verb!\euro!.%
\footnote{Das \euro\ Zeichen ist nicht im ursprünglichen LaTeX-Zeichensatz enthalten
sondern wird mit dem \texttt{eurosym}-Paket erzeugt.}


\subsection{Kommas in Dezimalzahlen (Mathematik-Modus)}

LaTeX\ setzt im Mathematik-Modus (also innerhalb von \verb!$$! oder in Gleichungen) nach dem angloamerikanischen Stil in Dezimalzahlen grundsätzlich den \emph{Punkt} (\verb!.!) als Trennsymbol voraus. So wird etwa mit \verb!$3.141$! normalerweise die Ausgabe "`3.141"' erzeugt. Um das in Europa übliche Komma in Dezimalzahlen zu verwenden, genügt es \emph{nicht}, einfach \verb!.! durch \verb!,! zu ersetzen. Das Komma wird in diesem Fall
als \textbf{Satzzeichen} interpretiert und sieht dann so aus:
\begin{quote}
\verb!$3,141$!	$\quad \rightarrow \quad 3,141$ 
\end{quote}
(man beachte den Leerraum nach dem Komma). Dieses Verhalten lässt sich in LaTeX\ zwar global umdefinieren, was aber wiederum zu einer Reihe unangenehmer Nebeneffekte führt. Eine einfache (wenn auch nicht sehr elegante) Lösung ist, Kommazahlen im Mathematik-Modus so zu schreiben:
\begin{quote}
\verb!$3{,}141$!	$\quad \rightarrow \quad 3{,}141$
\end{quote}



\subsection{Mathematische Werkzeuge}

Für die Erstellung komplizierter Gleichungen ist es mitunter
hilfreich, auf spezielle Software zurückzugreifen. Unter anderem kann man
aus dem Microsoft \emph{Equation Editor} und aus {\em
Mathematica} auf relativ einfache Weise LaTeX-An\-wei\-sun\-gen
für mathematische Gleichungen exportieren und direkt (mit etwas
manueller Nacharbeit) in das eigene LaTeX-Dokument übernehmen.


\subsection{Formelverzeichnis}

Jeder der mit Formeln arbeitet möchte auch ein Formelverzeichnis verwenden. Dies ist natürlich auch in diesem Template möglich und zwar über den Befehl \verb|\formelentry{Bezeichnung}|. Dieser kann entweder hinter die Formel geschrieben werden oder unter die verwendetet Umgebung \verb|\begin{align}Inhalt...\end{align}|. Hier ein Beispiel:

\begin{lstlisting}[style=LaTeX]
\begin{align}
A &= X + B\\
\formelentry{Formel 1} % Dies verlinkt auf die eine Formel
B &= X + C\\
\end{align}
\formelentry{Formel 2} % Dies verlinkt auf beide Formeln 
\end{lstlisting}

Die Verlinkung des Formelverzeichnisses ist auch für die Umgebung \emph{equation} möglich. Hier Beispiel dazu: 

\begin{lstlisting}[style=LaTeX]
\begin{equation}
C &= X + D\\
\formelentry{Formel 3}
\end{equation}
\end{lstlisting}

Die Idee für das Formelverzeichnis kommt aus diesem Blog \emph{qs-welt.de}:
\url{http://qs-welt.de/2014/05/01/formelverzeichnis-in-latex-erstellen/}

Weitere Möglichkeiten Verzeichnisse wie das Formelverzeichnis anzulegen erklärt Clemens Niederberger in seinem Blog: \url{http://www.mychemistry.eu/2017/04/define-new-floating-environment/}
%	\input{chapters/Schluss}
 
	\newpage

%----------------------------------------------------------------------------------------
%	BIBLIOGRAPHY und Listen
%----------------------------------------------------------------------------------------
	\begin{spacing}{1.0}
		\listoffigures
		\listoftables
		\lstlistoflistings
		\bibliography{bib/lib}
	%	\bibliographystyle{dinat} % Sortierung der Zitate nach Aufruf im Text
		\bibliographystyle{gerplain} % Deutscher Stil
		\nocite{*}
	\end{spacing}
	\newpage
	
%----------------------------------------------------------------------------------------
%	APPENDIX
%----------------------------------------------------------------------------------------
	\appendix
	\chapter{ToDo Liste}


Weitere Informationen zu Latex
\url{https://github.com/davidstutz/latex-resources}

\end{document}
