\pagestyle{fancy} %eigener Seitenstil
\fancyhf{} %alle Kopf- und Fußzeilenfelder bereinigen
\fancyhead[L]{} %Kopfzeile links
\fancyhead[C]{} %zentrierte Kopfzeile
\fancyhead[R]{} %Kopfzeile rechts
\renewcommand{\headrulewidth}{1.0pt} %obere Trennlinie
\fancyfoot[C]{\thepage} %Seitennummer
\renewcommand{\footrulewidth}{1.0pt} %untere Trennlinie

%\pagestyle{scrheadings}
%\ihead[]{\headmark}
%\chead[]{}
%\ohead[]{}
%\ifoot[]{}
%\cfoot[]{}
%\ofoot[\pagemark]{\pagemark}
%\setheadsepline{1.0pt}
%\setfootsepline{1.0pt}

\hypersetup{
	breaklinks = {true}, %Erlaubt Zeilenumbrüche in Links
	colorlinks = {true}, %Benutze farbige Links
	citecolor = {DocumentLinkColor}, %Farbe für Zitate
	linkcolor = {DocumentLinkColor}, %beeinflusst Inhaltsverzeichnis und Seitenzahlen
	urlcolor = {DocumentLinkColor}, %Weblink-Farbe
	pdftitle = {Konzeption und Entwicklung eines hochverfügbaren VPN-Clusters auf Basis von Debian Linux}, %Titel der Arbeit
	pdfsubject = {Bachelor-Thesis}, %Thema der Arbeit
	pdfauthor = {Jan Torben Bein}, %AutorIN der Arbeit
	pdfkeywords = {Cluster HA Linux Debian OpenVPN VPN Corosync Pacemaker Heartbeat}, %Stichwörter zur Arbeit
	pdfproducer = {pdfLaTeX}, %Erzeugt durch
	pdfcreator = {MiKTeX}, %Erstellt mit
	pdfstartview = {FitV},
	pdfview = {FitH},
	pdffitwindow = {true}
}
\def\UrlBreaks{\do\/\do-}

\setlength\parindent{0pt} % Removes all indentation from paragraphs

% Content and Stuff
\setcounter{tocdepth}{3}
\setcounter{secnumdepth}{3}

% Fonts and Style
\renewcommand*\familydefault{\sfdefault} %% Only if the base font of the document is to be sans serif

\addto\captionsngerman{
	\renewcommand{\figurename}{Abb.}
	\renewcommand{\tablename}{Tab.}
}

\setlength{\footskip}{2\baselineskip}

\fussy

% Für das Formelverzeichnis
\DeclareNewTOC[%
type=formel, 
name={Formel},
hang=5em,%
listname={Formelverzeichnis}
]{for}

\newcommand*{\formelentry}[1]{%
	\addcontentsline{for}{formel}{\protect\numberline{Formel~\theequation} #1}%
}