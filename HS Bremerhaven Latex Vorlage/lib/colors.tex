\definecolor{mygreen}{rgb}{0,0.6,0}
\definecolor{mygray}{rgb}{0.5,0.5,0.5}
\definecolor{mymauve}{rgb}{0.58,0,0.82}
\definecolor{TitleBox}{RGB}{177,177,237}
\definecolor{DocumentLinkColor}{rgb}{0,0,0}
\definecolor{background}{HTML}{eeeeee}
\definecolor{delim}{RGB}{20,105,176}
\colorlet{numb}{magenta!60!black}
\colorlet{punct}{red!60!black}

\lstdefinestyle{CodeBox} {
	language=C,
    basicstyle=\footnotesize\ttfamily,
    numbers=left,
%     numberstyle=\scriptsize,
    stepnumber=2,
    numbersep=8pt,
    showstringspaces=false,
    breaklines=true,
    frame=lines,
    tabsize=2,
    backgroundcolor=\color{background},
    breakatwhitespace=false,
	breaklines=true,
	commentstyle=\color{mygreen},
	deletekeywords={...},
	keepspaces=true,
	keywordstyle=\color{blue},
	morekeywords={*,...},
	numbers=left,
	numberstyle=\tiny\color{mygray},
	showspaces=false,
	showtabs=false,
	stringstyle=\color{mymauve}
}

\lstdefinestyle{ps1} {
	backgroundcolor=\color{white},   	% choose the background color; you must add \usepackage{color} or \usepackage{xcolor}
	basicstyle=\footnotesize\ttfamily, 	% the size of the fonts that are used for the code
	breakatwhitespace=false,        	% sets if automatic breaks should only happen at whitespace
	breaklines=true,               		% sets automatic line breaking
	commentstyle=\color{black},    		% comment style
	deletekeywords={...},            	% if you want to delete keywords from the given language
	escapeinside={\%*}{*)},          	% if you want to add LaTeX within your code
	extendedchars=true,              	% lets you use non-ASCII characters; for 8-bits encodings only, does not work with UTF-8
	frame=none,                  		% adds a frame-border to the bottom
	keepspaces=true,                 	% keeps spaces in text, useful for keeping indentation of code (possibly needs columns=flexible)
	keywordstyle=\color{black},       	% keyword style
	language=bash, 	                	% the language of the code
	morekeywords={*,...},            	% if you want to add more keywords to the set
	numbers=none,                    	% where to put the line-numbers; possible values are (none, left, right)
	showspaces=false,               	% show spaces everywhere adding particular underscores; it overrides 'showstringspaces'
	showstringspaces=false,          	% underline spaces within strings only
	showtabs=false,                  	% show tabs within strings adding particular underscores
	stringstyle=\color{black},     		% string literal style
	tabsize=4	                      	% sets default tabsize to 2 spaces
}

\lstdefinestyle{json}{
    basicstyle=\footnotesize\ttfamily,
    numbers=left,
    numberstyle=\scriptsize,
    stepnumber=2,
    numbersep=8pt,
    showstringspaces=false,
    breaklines=true,
    frame=lines,
    deletekeywords={...},
    backgroundcolor=\color{background},
    tabsize=2,
    literate=
     *{0}{{{\color{numb}0}}}{1}
      {1}{{{\color{numb}1}}}{1}
      {2}{{{\color{numb}2}}}{1}
      {3}{{{\color{numb}3}}}{1}
      {4}{{{\color{numb}4}}}{1}
      {5}{{{\color{numb}5}}}{1}
      {6}{{{\color{numb}6}}}{1}
      {7}{{{\color{numb}7}}}{1}
      {8}{{{\color{numb}8}}}{1}
      {9}{{{\color{numb}9}}}{1}
      {:}{{{\color{punct}{:}}}}{1}
      {,}{{{\color{punct}{,}}}}{1}
      {\{}{{{\color{delim}{\{}}}}{1}
      {\}}{{{\color{delim}{\}}}}}{1}
      {[}{{{\color{delim}{[}}}}{1}
      {]}{{{\color{delim}{]}}}}{1},
}

% \lstset{style=CodeBox}
\DeclareCaptionFont{white}{\color{white}}
% \DeclareCaptionFormat{listing}{\colorbox{TitleBox}{\parbox{\fboxsep}{#1#2#3}}}
\DeclareCaptionFormat{listing}{\colorbox{TitleBox}{\parbox{\dimexpr\textwidth-2\fboxsep\relax}{#1#2#3}}}
\captionsetup[lstlisting] {
	format=listing,
	font={sf}
}