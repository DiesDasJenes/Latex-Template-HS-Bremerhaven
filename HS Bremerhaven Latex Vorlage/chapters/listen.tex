\chapter{Listen}

Um eine Liste zu erstellen können die Umgebungen itemize, enumerate und description in \verb|\begin{}\end{}| genutzt werden. Hier ein Link zu Beispielen: \url{https://www.namsu.de/Extra/befehle/Auflistungen.html} \newline

Die itemize Umgebung der Listen erstellt die Aufzählungszeichen vor den "'Items"'. Diese Zeichen können über \verb|\item[Symbol]| manuell angepasst werden. Dies kann auch durch \LaTeX umgesetzt werden. Dazu sollte das Paket "'Paralist"' verwendet werden.  
\begin{itemize}
	\item Ein Item mit dem Standard Punkt
	\item[->] Ein Item mit Pfeil
	\begin{enumerate}
		\item Ein Enumerate Liste innerhalb einer itemize Liste
		\item Es wird nummeriert
		\begin{description}
			\item[Eins] In der description Liste wird der Punkt durch Text ersetzt
			\item[Zwei] Dies ist sehr hilfreich bei Aufzählungen, welche weiterer Erklärung benötigen.
		\end{description}
	\end{enumerate}
\end{itemize}

Das Paket \emph{Paralist} enthält einige neue Listenumgebungen. Einzel- und
Aufzählungslisten können innerhalb von Absätzen gesetzt werden, sowie als Absätze und
in kompakter Ausführung. Die meisten Umgebungen haben optionale Argumente
um die Aufzählungszeichen zu formatieren. Zusätzlich können die LATEX-Umgebungen aufgeschlüsselt nach itemize und enumerate erweitert werden, um ein ähnliches optionales Argument zu verwenden. \newline

Hier ein paar Beispiele:

\begin{enumerate}[(i)]
	\item Dadurch kann die Umgebung die automatische weiterzählung von Listenpunkten
	\item Ich bin Item Nummer 2
\end{enumerate} 

oder 

\begin{enumerate}[{Beispiel} (a)]
	\item Es muss also nicht selbst die Liste nummeriert werden
	\item Auch hier geschieht es automatisch
\end{enumerate}

Listen mit weniger Platz zwischen den Zeilen (compactitem)
Manchmal kann es vorkommen das der Abstand zwischen den Listenelementen zu groß für eine Seite ist. Um den Abstand zwischen den Listenelementen zu verkleinern kann der Befehl \verb|\begin{compactitem}\end{compactitem}| oder \emph{compactenum} verwendet werden.
 
\begin{compactitem}
	\item Dies ist ein Beispiel
	\item das Listenelemente nicht so viel Platz verbrauchen müssen
\end{compactitem}

\vspace{1cm} %Um den Zwischenraum zu erzeugen wurde \vspace genutzt. Es können cm oder pt als Wert eingesetzt werden
Um Listen nebeneinander zu platzieren, muss auf die Tabellen Umgebung von \LaTeX zurückgegriffen werden. 

\begin{tabular}{ll}
	\parbox{5cm}{
		\begin{itemize}
			\item Teil 1
			\item Teil 2
	\end{itemize}}
	&
	\parbox{5cm}{
		\begin{itemize}
			\item Teil 3
			\item Teil 4
	\end{itemize}}
\end{tabular}