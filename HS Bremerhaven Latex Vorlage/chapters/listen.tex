\chapter{Listen}

Um eine Liste erstellen zu können, werden die Umgebungen itemize, enumerate und description in \verb|\begin{}\end{}| genutzt. Hier ein Link zu Beispielen: \url{https://www.namsu.de/Extra/befehle/Auflistungen.html} \newline

Die itemize Umgebung der Listen erstellt die Aufzählungszeichen vor den "'Items"'. Diese Zeichen können über \verb|\item[Symbol]| manuell angepasst werden. An dieser Stelle sei das Paket "'Paralist"' erwähnt, welches die Listen Umgebungen um weitere Aufzählungsvorlagen erweitert und auch diese kompakter darstellen lässt.  
\begin{itemize}
	\item Ein Item mit dem Standardpunkt
	\item[->] Ein Item mit Pfeil
	\begin{enumerate}
		\item Ein Enumerate Liste innerhalb einer itemize Liste
		\item Es wird nummeriert
		\begin{description}
			\item[Eins] In der description Liste wird der Punkt durch Text ersetzt
			\item[Zwei] Dies ist sehr hilfreich bei Aufzählungen, welche weiterer Erklärung benötigen.
		\end{description}
	\end{enumerate}
\end{itemize}

Das Paket \emph{Paralist} enthält einige neue Listenumgebungen. Einzel- und
Aufzählungslisten können innerhalb von Absätzen gesetzt werden. Die meisten Umgebungen haben optionale Argumente
um die Aufzählungszeichen zu formatieren. Zusätzlich können die LATEX-Umgebungen aufgeschlüsselt nach itemize und enumerate erweitert werden, um ein ähnliches optionales Argument zu verwenden. \newline

Hier ein paar Beispiele:

\begin{enumerate}[(i)]
	\item Dadurch kann die Umgebung die automatische Weiterzählung von Listenpunkten
	\item Ich bin Item Nummer 2
\end{enumerate}

oder 

\begin{enumerate}[{Beispiel} (a)]
	\item Es muss also nicht selbst die Liste nummeriert werden
	\item Auch hier geschieht es automatisch
\end{enumerate}

\textbf{Listen mit weniger Platz zwischen den Zeilen (compactitem)}

Manchmal kann es vorkommen, dass der Abstand zwischen den Listenelementen zu groß ist für eine Seite. Um den Abstand zwischen den Listenelementen zu verkleinern kann der Befehl \verb|\begin{compactitem}\end{compactitem}| oder \emph{compactenum} verwendet werden.
 
\begin{compactitem}
	\item Dieser Beispieltext in dieser Liste zeigt
	\item dass Listenelemente nicht so viel Platz verbrauchen müssen
\end{compactitem}

\vspace{1cm} %Um den Zwischenraum zu erzeugen wurde \vspace genutzt. Es können cm oder pt als Wert eingesetzt werden
Um Listen nebeneinander zu platzieren, muss auf Tabellen in \LaTeX{} zurückgegriffen werden. 

\begin{tabular}{ll}
	\parbox{5cm}{
		\begin{itemize}
			\item Teil 1
			\item Teil 2
	\end{itemize}}
	&
	\parbox{5cm}{
		\begin{itemize}
			\item Teil 3
			\item Teil 4
	\end{itemize}}
\end{tabular}