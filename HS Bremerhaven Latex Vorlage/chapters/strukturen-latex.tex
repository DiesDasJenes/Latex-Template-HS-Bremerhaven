\chapter{Strukturen in Latex}

In Latex werden Kapitel mit \verb+\chapter{}+ erstellt. Ein Abschnitt wird mit \verb+\section{}+ und einen Absatz mit \verb+\paragraph+ definiert. Sie können auch einen Unterabschnitt mit \verb+\subsection{}+ und einen Unterabsatz mit \verb+\subparagraph{}+ hinzufügen.\\

Es gibt ebenso noch die \verb|\\|, \verb|\\\\| sowie \verb|\newline| um einen Text um eine Zeile zu verrücken. \newline

Sie sollten Unterabschnitte stets nur bis zur zweiten Ebene unterteilen. Manchmal wird auch die dritte Ebene erlaubt. Dies ist vom Professor abhängig. Wenn es dennoch notwendig sein sollte, eine höhere Anzahl an Unterabschnitten hinzuzufügen, zum Beispiel bis in die 100te Ebene, hilft folgender Link dabei: \url{https://tex.stackexchange.com/questions/30997/more-section-headings}.\newline



Die Umgebung \verb|\begin{quote}\end{quote}| sollte genutzt werden um Zitate im Text einfließen zu lassen. Es können aber auch die Anführungszeichen \verb|"'Text"'| genutzt werden.

\begin{quote}
	Der Mensch ist immer noch der beste Computer.\\
	John F. Kennedy \autocite{jfkey}
\end{quote}
