\chapter{Einleitung}
Willkommen zur Einführung in die Vorlage für eure schriftlichen Leistungsnachweise bzw. Bachelor- oder Masterbschlussarbeiten. Wir werden gemeinsam den Inhalt der Latex Vorlage durchgehen und dabei die Vorteile und Möglichkeiten von Latex ergründen. Beginnen wir mit der Dateistruktur dieser Vorlage.

\begin{itemize}
	\item Vorlage/
		\begin{itemize}
			\item bib/ - Unter diesem Ordner findet ihr die Bibtex Datei welche ihr mit Programmen wie readcube oder Jabref bearbeiten könnt. \url{http://www.jabref.org/}. Außerdem interessant für alle \url{https://www.mendeley.com/} und \url{http://www.citeulike.org/}
			\item chapter/ - In diesem Ordner werden sämtliche Latex .tex Dateien abgelegt in denen ihr euren Text verfasst. Ein Tipp: verteilt die Kapitel und Sektionen auf verschiedene Dateien. Dies ermöglicht ein paralleles Arbeiten und erleichtert die Arbeit an längeren Texten.
			\item images/ - Wie der Name schon sagt, ist dies der Ordner für Bilder aller Art. Benutzt dabei am besten .png, .svg oder .pdf da diese Dateiformate verlustfrei beim skalieren der Größe sind.
			\item lib/ - Hier sind sämtliche Dateien abgelegt mit denen das Aussehen des Dokumentes verändert werden kann. 
		\end{itemize}
\end{itemize}

Wenn ihr \textbf{komplette Neulinge} in LaTeX seid schaut gleichzeitig hier vorbei: \\\url{http://latex.tugraz.at/latex/tutorial}\\
\url{https://github.com/VoLuong/Begin-Latex-in-minutes}\\
http://texwelt.de/wissen/\\
Für alle die im Browser kollaborativ arbeiten möchten empfiehlt es sich \url{www.overleaf.com} zu nutzen. Wenn man lieber auf seinem PC arbeiten möchte, ist unter Windows die Tex-Umgebung \url{miktex.org} zu installieren. Auf dem Betriebssystem Linux muss in einem Terminal folgender Befehl eingegeben werden: "'sudo apt-get install texlive-full"'. Damit werden sämtliche Pakete und Sprachen heruntergeladen. Auf beiden Betriebssystemen bietet es sich an den Tex-Editor \emph{TexStudio} zu nutzen. Es gibt jedoch noch andere: \url{https://en.wikipedia.org/wiki/Comparison_of_TeX_editors}.

Bei weiteren Fragen oder Problemem zur Installation von Tex hier ein Link:\\
\url{http://texwelt.de/wissen/fragen/11038/wie-installiere-ich-latex}

\section{Wie beginnt man?}

Ein jedes Dokument beginnt man damit, die Metadaten zu bearbeiten und das Titelblatt einzustellen. In der Index TeX Datei "'Index.tex"' findet man dazu unter dem Abschnitt TITLE-Page zwei \textbf{input} Befehle die zu den Seiten \textbf{titlepage\_alone} und \textbf{titlepage\_group} verweisen. Wie die Namen schon aussagen ist jeder der Seiten für eine Sache optimiert. Bei einer Gruppe ab drei Personen wird empfohlen die Gruppenversion zu nehmen. Die Auswahl ist natürlich jedem selbst überlassen.\\

Weiter geht es zu den Metadaten in der Tex-Datei \textbf{person-configuration} in dem Ordner \emph{lib/}. Dort sollten sämtliche Informationen hinter den Befehlen angepasst werden. Die ersten Veränderungen können dann bereits auf dem Deckblatt verzeichnet werden.


\section{Aufbau der Index Datei}

Die Index-Datei beinhaltet die Struktur des Dokumentes. Sie ist eingeteilt in die Abschnitte:
\begin{itemize}
	\item Preamble: Hier werden die Einstellungen für das 'scrbook' übergeben. Wenn ihr eine anderes Aussehen möchtet, probiert stattdessen 'article'
	\item Einstellungen: Hier werden die Einstellungen zu Packages, Nutzerinformationen und Aussehen der Seiten geladen
	\item Bereich des Dokumentes: In diesem Segment wird die Abfolge des Dokumentes festgelegt. 
	\begin{itemize}
		\item Title-Page: Es stehen zwei Arten von Deckblättern zur Verfügung. Für Einzel- und Gruppenarbeiten.
		\item ABSTRACT: Auf dieser Seite muss das Thema des Leistungsnachweises kurz und bündig beschrieben werden.
		\item Explanation: In der Erklärung weist man darauf hin, dass kein Plagiat angefertigt wurde. Vergesst nicht zu unterschreiben bevor ihr abgebt.
		\item TABLE OF CONTENTS: Unter diesem Abschnitt wird festgelegt in welcher Reihenfolge die Gliederungen geordnet werden. Dabei ist interessant, dass mit dem Befehl \emph{pagenumbering} die Nummerierung auf "'Roman"' gesetzt wird und mit \emph{setcounter} die Zählung bei 1 zurückgesetzt wird.
		\item GLOSSAR: Hier sollten Fremdbegriffe welche im Dokument vorkommen mit kurzer Definition erklärt werden, sodass die Abkürzungen im Dokument benutzt werden kann.
		\item CHAPTERS: Hier bindet ihr eure Kapitel und Abschnitte ein.
		\item BIBLIOGRAPHY: Literaturverzeichnis wird am besten mit Jabref gepflegt
		\item APPENDIX: Wenn ihr einen Anhang benötigt für größere Bilder oder Diagramme, ist hier der richtige Ort dafür.
	\end{itemize}
\end{itemize}


\section{Pakete und CTAN}

Im Umgang mit LaTex werden sie sehr schnell auf das Problem stoßen das sie etwas tun möchten was von den in diesem Template vorhandenen Paketen nicht unterstützt wird. Um die Funktionalität hinzuzufügen suchen sie über eine Suchplattform (DuckDuckGo, Google etc.) ihrer Wahl nach dem richtigen \LaTeX Paket. In der Datei "'/lib/packages.tex"' sollten sie das Paket mittels des Befehls \verb|\usepackage{package}| einbinden. Sie können diesen Befehl überall im Dokument platzieren, der Übersicht halber sollte dies jedoch in der "'/lib/packages.tex"' geschehen. Nachdem das Paket eingebunden ist werden sie beim nächsten Compilieren ihres Dokumentes gefragt ob das Paket heruntergeladen werden soll. Bei der Installation haben sie bereits ein Repository einer unabhängigen Universität oder Hochschule ausgewählt. Jedes Paket hat ein Dokumentation, welche auf der Webseite \url{https://ctan.org} zu finden ist. Es empfiehlt sich stets die Dokumentation zu lesen, da diese meistens jede aufkommende Frage beantwortet. 