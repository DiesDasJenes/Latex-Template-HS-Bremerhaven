\chapter{Tabellen}

Tabellen werden in der zugehörigen Umgebung geschrieben \verb|\begin{tabular}\end{tabular}|
\begin{lstlisting}[style=Latex,caption={Einfache Tabelle},label=lst:tab1]
\begin{tabular}{|lcr||}
left aligned column & center column & right column \\
\hline
text & text & text \\
text & text & text \\
\end{tabular}
\end{lstlisting}

\begin{tabular}{|lcr||}
	left aligned column & center column & right column \\
	\hline
	text & text & text \\
	text & text & text \\
\end{tabular}
%Skipping for 1 row there is also \smallskip and \medskip
\bigskip

Der Parameter (im Beispiel |lcr|||) wird als Tabellenspezifikation bezeichnet und teilt \LaTeX{} mit, wie viele Spalten es dort gibt und wie diese formatiert werden sollen. Jeder Buchstabe steht dabei für eine einzelne Spalte. Mögliche Werte sind:

\begin{table}[ht]
	\centering
	\begin{tabular}{ll}
		\multicolumn{1}{c}{\textbf{Zeichen}} & \multicolumn{1}{c}{\textbf{Bedeutung}}       \\
		l                           & linksbündige Spalte                 \\
		c                           & zentrierte Spalte                   \\
		r                           & rechtsbündige Spalte                \\
		p\{'width'\} e.g. p\{5cm\}  & Absatzspalte mit definierter Breite \\
		| (pipe character)          & vertikale Linie                     \\
		|| (2 pipes)                & 2 vertikale Linien                 
	\end{tabular}
\end{table}

Zellen werden durch das Zeichen \& getrennt. Eine Reihe wird durch \textbackslash\textbackslash Back Slashes beendet und horizontale Linien können mit dem Befehl \verb|\hline| eingefügt werden.
Tabellen sind durch \LaTeX{} in ihrer Breite stets so skaliert, dass den gesamten Inhalt enthalten. Wenn eine Tabelle zu groß ist, druckt \LaTeX{} überfüllte hbox Warnungen. Mögliche Lösungen sind die Verwendung des p{'width'} Parameters oder anderer Pakete wie \emph{tabularx}. Eine Tabelle mit Spaltenüberschriften, die sich über mehrere Spalten erstrecken, kann mit dem Befehl \verb|\multicolumn{cols}{pos}{text}| realisiert werden.

\begin{lstlisting}[style=Latex,caption={Einfache Tabelle mit Multicolumn},label=lst:tab2]
\begin{center}
\begin{tabular}{|c|c|c|c|}
\hline
&\multicolumn{3}{|c|}{Income Groups}\\
\cline{2-4}
City&Lower&Middle&Higher\\
\hline
City-1& 11 & 21 & 13\\
City-2& 21 & 31 &41\\
\hline
\end{tabular}
\end{center}
\end{lstlisting}

\begin{center}
	\begin{tabular}{|c|c|c|c|}
		\hline
		&\multicolumn{3}{|c|}{Income Groups}\\
		\cline{2-4}
		City&Lower&Middle&Higher\\
		\hline
		City-1& 11 & 21 & 13\\
		City-2& 21 & 31 &41\\
		\hline
	\end{tabular}
\end{center}
\bigskip

Beachtet, dass der Befehl \verb|\multicolumn| drei obligatorische Argumente enthält: Das erste Argument gibt die Anzahl der Spalten, über die sich die Überschrift erstreckt, das zweite Argument spezifiziert die Position der Überschrift (l,c,r) und das dritte Argument ist der Text für die Überschrift. Eingesetzt wurde dieser Befehl bereits bei der Tabelle darüber, um die Überschriften in den Zeilen mittig einzurücken. Der Befehl \verb|\cline{2-4}| spezifiziert die Startspalte (hier, 2) und Endspalte (hier, 4), über die eine Linie gezogen werden soll.

\bigskip

Um die Tabelle innerhalb der Seite anzuordnen, wird noch dazu die Umgebung \emph{table} benötigt. \LaTeX{} platziert eine Tabelle mit der Umgebung \verb|\begin{table}[parameter]\end{table}| anhand verschiedener interner Regeln automatisch an einer vorteilhaften Position innerhalb des Textes. Der Parameter \textbf{b} steht für den unteren Rand einer Seite, \textbf{t} für den oberen Rand, \textbf{h} für die Stelle an der die Tabelle im Quellcode definiert wurde und \textbf{p} bedeutet, dass die Tabelle auf einer eigenen Seite platziert wird (ggf. zusammen mit weiteren Tabellen). Die *-Variante der Umgebung sorgt bei zweispaltigem Textsatz dafür, dass die Tabelle über beide Spalten hinweg Platz einnimmt.

\begin{lstlisting}[style=Latex,caption={Einfache Tabelle mit Umgebung table},label=lst:tab3]
\begin{table}[ht]
\centering
\begin{tabular}{|lcr||}
left aligned column & center column & right column \\
\hline
text & text & text \\
text & text & text \\
\end{tabular}
\end{table}
\end{lstlisting}

Lange Tabellen werden von \LaTeX{} nativ unterstützt, dank der Langzeitumgebung. Leider unterstützt diese Umgebung kein Stretching (X-Spalten) über mehrere Seiten.

Die Pakete von tabular stellen die Longtabu-Umgebung zur Verfügung. Es hat die meisten Eigenschaften von Tabu, mit der zusätzlichen Fähigkeit, mehrere Seiten zu überspannen.

\LaTeX{} kann mit langen Tabellen gut umgehen: Sie können eine Kopfzeile angeben, die sich auf jeder Seite wiederholt, eine Kopfzeile nur für die erste Seite und die gleiche für die Fußzeile.

Es verwendet eine Syntax, die dem Paket \emph{Longtable} ähnelt, daher sollten Sie einen Blick in die Dokumentation werfen, wenn Sie mehr darüber erfahren möchten.

Alternativ können Sie auch eines der folgenden Pakete \emph{supertabular} oder \emph{xtab} ausprobieren, eine erweiterte und etwas verbesserte Version von \emph{supertabular}.
Folgendes Beispiel stammt von \url{http://users.sdsc.edu/~ssmallen/latex/longtable.html}

\bigskip
\begin{lstlisting}[style=Latex,caption={LongTable Beispiel für Übergroße Tabellen},label=lst:tab4]
\begin{center}
\begin{longtable}{|l|l|l|}
\caption[Feasible triples for a highly variable Grid]{Feasible triples for 
highly variable Grid, MLMMH.} \label{grid_mlmmh} \\

\hline \multicolumn{1}{|c|}{\textbf{Time (s)}} & \multicolumn{1}{c|}{\textbf{Triple chosen}} & \multicolumn{1}{c|}{\textbf{Other feasible triples}} \\ \hline 
\endfirsthead

\multicolumn{3}{c}%
{{\bfseries \tablename\ \thetable{} -- continued from previous page}} \\
\hline \multicolumn{1}{|c|}{\textbf{Time (s)}} &
\multicolumn{1}{c|}{\textbf{Triple chosen}} &
\multicolumn{1}{c|}{\textbf{Other feasible triples}} \\ \hline 
\endhead

\hline \multicolumn{3}{|r|}{{Continued on next page}} \\ \hline
\endfoot

\hline \hline
\endlastfoot

0 & (1, 11, 13725) & (1, 12, 10980), (1, 13, 8235), (2, 2, 0), (3, 1, 0) \\
2745 & (1, 12, 10980) & (1, 13, 8235), (2, 2, 0), (2, 3, 0), (3, 1, 0) \\
5490 & (1, 12, 13725) & (2, 2, 2745), (2, 3, 0), (3, 1, 0) \\
8235 & (1, 12, 16470) & (1, 13, 13725), (2, 2, 2745), (2, 3, 0), (3, 1, 0) \\
10980 & (1, 12, 16470) & (1, 13, 13725), (2, 2, 2745), (2, 3, 0), (3, 1, 0) \\
13725 & (1, 12, 16470) & (1, 13, 13725), (2, 2, 2745), (2, 3, 0), (3, 1, 0) \\
.........
\end{lstlisting}

\begin{center}
	\begin{longtable}{|l|l|l|}
		\caption[Feasible triples for a highly variable Grid]{Feasible triples for 
			highly variable Grid, MLMMH.} \label{grid_mlmmh} \\
		
		\hline \multicolumn{1}{|c|}{\textbf{Time (s)}} & \multicolumn{1}{c|}{\textbf{Triple chosen}} & \multicolumn{1}{c|}{\textbf{Other feasible triples}} \\ \hline 
		\endfirsthead
		
		\multicolumn{3}{c}%
		{{\bfseries \tablename\ \thetable{} -- Fortsetzung von vorheriger Seite}} \\
		\hline \multicolumn{1}{|c|}{\textbf{Time (s)}} &
		\multicolumn{1}{c|}{\textbf{Triple chosen}} &
		\multicolumn{1}{c|}{\textbf{Other feasible triples}} \\ \hline 
		\endhead
		
		\hline \multicolumn{3}{|r|}{{Fortsetzung auf der nächsten Seite}} \\ \hline
		\endfoot
		
		\hline \hline
		\endlastfoot
		
		0 & (1, 11, 13725) & (1, 12, 10980), (1, 13, 8235), (2, 2, 0), (3, 1, 0) \\
		2745 & (1, 12, 10980) & (1, 13, 8235), (2, 2, 0), (2, 3, 0), (3, 1, 0) \\
		5490 & (1, 12, 13725) & (2, 2, 2745), (2, 3, 0), (3, 1, 0) \\
		8235 & (1, 12, 16470) & (1, 13, 13725), (2, 2, 2745), (2, 3, 0), (3, 1, 0) \\
		10980 & (1, 12, 16470) & (1, 13, 13725), (2, 2, 2745), (2, 3, 0), (3, 1, 0) \\
		13725 & (1, 12, 16470) & (1, 13, 13725), (2, 2, 2745), (2, 3, 0), (3, 1, 0) \\
		16470 & (1, 13, 16470) & (2, 2, 2745), (2, 3, 0), (3, 1, 0) \\
		19215 & (1, 12, 16470) & (1, 13, 13725), (2, 2, 2745), (2, 3, 0), (3, 1, 0) \\
		21960 & (1, 12, 16470) & (1, 13, 13725), (2, 2, 2745), (2, 3, 0), (3, 1, 0) \\
		24705 & (1, 12, 16470) & (1, 13, 13725), (2, 2, 2745), (2, 3, 0), (3, 1, 0) \\
		27450 & (1, 12, 16470) & (1, 13, 13725), (2, 2, 2745), (2, 3, 0), (3, 1, 0) \\
		30195 & (2, 2, 2745) & (2, 3, 0), (3, 1, 0) \\
		32940 & (1, 13, 16470) & (2, 2, 2745), (2, 3, 0), (3, 1, 0) \\
		35685 & (1, 13, 13725) & (2, 2, 2745), (2, 3, 0), (3, 1, 0) \\
		38430 & (1, 13, 10980) & (2, 2, 2745), (2, 3, 0), (3, 1, 0) \\
		41175 & (1, 12, 13725) & (1, 13, 10980), (2, 2, 2745), (2, 3, 0), (3, 1, 0) \\
		43920 & (1, 13, 10980) & (2, 2, 2745), (2, 3, 0), (3, 1, 0) \\
		46665 & (2, 2, 2745) & (2, 3, 0), (3, 1, 0) \\
		49410 & (2, 2, 2745) & (2, 3, 0), (3, 1, 0) \\
		52155 & (1, 12, 16470) & (1, 13, 13725), (2, 2, 2745), (2, 3, 0), (3, 1, 0) \\
		54900 & (1, 13, 13725) & (2, 2, 2745), (2, 3, 0), (3, 1, 0) \\
		57645 & (1, 13, 13725) & (2, 2, 2745), (2, 3, 0), (3, 1, 0) \\
		60390 & (1, 12, 13725) & (2, 2, 2745), (2, 3, 0), (3, 1, 0) \\
		63135 & (1, 13, 16470) & (2, 2, 2745), (2, 3, 0), (3, 1, 0) \\
		65880 & (1, 13, 16470) & (2, 2, 2745), (2, 3, 0), (3, 1, 0) \\
		68625 & (2, 2, 2745) & (2, 3, 0), (3, 1, 0) \\
		71370 & (1, 13, 13725) & (2, 2, 2745), (2, 3, 0), (3, 1, 0) \\
		74115 & (1, 12, 13725) & (2, 2, 2745), (2, 3, 0), (3, 1, 0) \\
		76860 & (1, 13, 13725) & (2, 2, 2745), (2, 3, 0), (3, 1, 0) \\
		79605 & (1, 13, 13725) & (2, 2, 2745), (2, 3, 0), (3, 1, 0) \\
		82350 & (1, 12, 13725) & (2, 2, 2745), (2, 3, 0), (3, 1, 0) \\
		85095 & (1, 12, 13725) & (1, 13, 10980), (2, 2, 2745), (2, 3, 0), (3, 1, 0) \\
		87840 & (1, 13, 16470) & (2, 2, 2745), (2, 3, 0), (3, 1, 0) \\
		90585 & (1, 13, 16470) & (2, 2, 2745), (2, 3, 0), (3, 1, 0) \\
		93330 & (1, 13, 13725) & (2, 2, 2745), (2, 3, 0), (3, 1, 0) \\
		96075 & (1, 13, 16470) & (2, 2, 2745), (2, 3, 0), (3, 1, 0) \\
		98820 & (1, 13, 16470) & (2, 2, 2745), (2, 3, 0), (3, 1, 0) \\
		101565 & (1, 13, 13725) & (2, 2, 2745), (2, 3, 0), (3, 1, 0) \\
		104310 & (1, 13, 16470) & (2, 2, 2745), (2, 3, 0), (3, 1, 0) \\
		107055 & (1, 13, 13725) & (2, 2, 2745), (2, 3, 0), (3, 1, 0) \\
		109800 & (1, 13, 13725) & (2, 2, 2745), (2, 3, 0), (3, 1, 0) \\
		112545 & (1, 12, 16470) & (1, 13, 13725), (2, 2, 2745), (2, 3, 0), (3, 1, 0) \\
		115290 & (1, 13, 16470) & (2, 2, 2745), (2, 3, 0), (3, 1, 0) \\
		118035 & (1, 13, 13725) & (2, 2, 2745), (2, 3, 0), (3, 1, 0) \\
		120780 & (1, 13, 16470) & (2, 2, 2745), (2, 3, 0), (3, 1, 0) \\
		123525 & (1, 13, 13725) & (2, 2, 2745), (2, 3, 0), (3, 1, 0) \\
		126270 & (1, 12, 16470) & (1, 13, 13725), (2, 2, 2745), (2, 3, 0), (3, 1, 0) \\
		129015 & (2, 2, 2745) & (2, 3, 0), (3, 1, 0) \\
		131760 & (2, 2, 2745) & (2, 3, 0), (3, 1, 0) \\
		134505 & (1, 13, 16470) & (2, 2, 2745), (2, 3, 0), (3, 1, 0) \\
		137250 & (1, 13, 13725) & (2, 2, 2745), (2, 3, 0), (3, 1, 0) \\
		139995 & (2, 2, 2745) & (2, 3, 0), (3, 1, 0) \\
		142740 & (2, 2, 2745) & (2, 3, 0), (3, 1, 0) \\
		145485 & (1, 12, 16470) & (1, 13, 13725), (2, 2, 2745), (2, 3, 0), (3, 1, 0) \\
		148230 & (2, 2, 2745) & (2, 3, 0), (3, 1, 0) \\
		150975 & (1, 13, 16470) & (2, 2, 2745), (2, 3, 0), (3, 1, 0) \\
		153720 & (1, 12, 13725) & (2, 2, 2745), (2, 3, 0), (3, 1, 0) \\
		156465 & (1, 13, 13725) & (2, 2, 2745), (2, 3, 0), (3, 1, 0) \\
		159210 & (1, 13, 13725) & (2, 2, 2745), (2, 3, 0), (3, 1, 0) \\
		161955 & (1, 13, 16470) & (2, 2, 2745), (2, 3, 0), (3, 1, 0) \\
		164700 & (1, 13, 13725) & (2, 2, 2745), (2, 3, 0), (3, 1, 0) \\
	\end{longtable}
\end{center}

Hier noch einmal das Beispiel für die Umgebung \emph{longtabu}.
\begin{lstlisting}[style=Latex,caption={Beispiel zur Umgebung: LongTabu},label=lst:tab5]
\begin{longtabu} to \linewidth {lX[2]lXl}
\rowfont\bfseries H1 & H2 & H3 & H4 & H5 \\ \hline 
\endhead
\\ \hline
\multicolumn{5}{r}{There is more to come} \\
\endfoot
\\ \hline
\endlastfoot
% Inhalt ...
\end{lstlisting}


\textbf{Einfärben von Zellen/Zeilen und Spalten}\\

Um die Lesbarkeit einer Tabelle zu erhöhen können folgende Elemente eingefärbt werden: 
\begin{itemize}
	\item Zeilen
	\item Spalten
	\item Linien
	\item Zellen
\end{itemize}

\textbf{Einfärben von Zeilen}

Dazu wird der Befehl \verb|\rowcolor| benutzt, welcher durch das Paket \emph{xcolor} bereitgestellt wird.

\begin{lstlisting}[style=Latex,caption={Einfache Tabelle mit Einfärbung der Zeile},label=lst:tab6]
\documentclass{article}
\usepackage[table]{xcolor}
\begin{document}
\begin{tabular}{ | l | l | l | }
\rowcolor{green}
A & B & C \\
\rowcolor{red}
D & E & F \\
G & H & I \\
\rowcolor{blue}
J & K & L
\end{tabular}
\end{document}
\end{lstlisting}


\begin{tabular}{ | l | l | l | }
	\rowcolor{green}
	A & B & C \\
	\rowcolor{red}
	D & E & F \\
	G & H & I \\
	\rowcolor{blue}
	J & K & L
\end{tabular}

\bigskip % Für den Abstand zur Tabelle

\textbf{Einfärben von Spalten}

Spalten können auf zwei Wegen eingefärbt werden. Der erste Weg wäre die Farbeigenschaft in den Tabellenparametern zu definieren. Der andere Weg ist dafür eine sogenannte \verb|\newcolumntype{name}| festzulegen.

\begin{lstlisting}[style=Latex,caption={Einfache Tabelle mit Einfärbung der Spalte},label=lst:tab7]
\documentclass{article}
\usepackage[table]{xcolor}
% Weg 2 ColumnType anlegen
\newcolumntype{a}{>{\columncolor{yellow}}c}
\newcolumntype{b}{>{\columncolor{green}}c}
\begin{document}
% Weg 1 Farbe in den Parametern der Tabelle definieren
\begin{tabular}{ a | >{\columncolor{red}}c | l | b }
\hline
A & B & C & D \\
E & F & G & H \\
\hline
\end{tabular}
\end{document}
\end{lstlisting}
 
 
 % Weg 2 ColumnType anlegen
 \newcolumntype{a}{>{\columncolor{yellow}}c}
 \newcolumntype{b}{>{\columncolor{green}}c}
 % Weg 1 Farbe in den Parametern der Tabelle definieren
 \begin{tabular}{ a | >{\columncolor{red}}c | l | b }
 	\hline
 	A & B & C & D \\
 	E & F & G & H \\
 	\hline
 \end{tabular}

\bigskip

\textbf{Einfärben von Linien}

Über den Befehl \verb|\arrayrulecolor{text}| können Linien einer ganzen Tabelle eingefärbt werden. Wenn man nur eine einfärben möchte, muss unter der Tabelle die Farbe wieder auf Schwarz gestellt werden. Es ist auch möglich nur einzelne Zeilen einzufärben, indem ein neuer Befehl definiert wird:\\ \verb|\newcommand\tln[1]{\arrayrulecolor{#1}\hline}|.

\begin{lstlisting}[style=Latex,caption={Einfache Tabelle mit Einfärbung der Linien},label=lst:tab8]
\documentclass{article}
\usepackage[table]{xcolor}
\arrayrulecolor{blue}
\begin{document}
\begin{tabular}{ | l | l | l | }
\hline
A & B & C \\
\hline
D & E & F\\
\hline
G & H & I \\
\hline
\end{tabular}
\end{document}
\end{lstlisting}

\arrayrulecolor{blue}
	\begin{tabular}{ | l | l | l | }
		\hline
		A & B & C \\
		\hline
		D & E & F\\
		\hline
		G & H & I \\
		\hline
	\end{tabular}

Hier das Beispiel für die Zeilenweise Einfärbung.
\newcommand\tln[1]{\arrayrulecolor{#1}\hline}
\arrayrulecolor{black}
\begin{tabular}{ | l | l | l | }
	\hline
	A & B & C \\ 
	\tln{red}
	D & E & F \\ 
	\hline
	G & H & I \\
	\hline
\end{tabular}
\arrayrulecolor{black}

\bigskip

\textbf{Einfärben von einzelnen Zellen}

Um einzelne Zellen einzufärben, bedarf es dem Befehl \verb|\cellcolor{color}|, welcher ebenfalls durch das Paket \emph{xcolor} bereitgestellt wird. \LaTeX{} erlaubt es zudem über den Befehl \verb|\definecolor{Gray}{gray}{0.85}| oder\\ \verb|\columncolor[RGB oder HTML]{232, 232, 122 oder AAACED}| Farben zu definieren. 

\begin{lstlisting}[style=Latex,caption={Einfache Tabelle mit Einfärbung von Zellen},label=lst:tab9]
\documentclass{article}
\usepackage[table]{xcolor}
\arrayrulecolor{blue}
\begin{document}
\begin{tabular}{ | l | l | l | }
\hline
A & B & C \\
\hline
D & E & \cellcolor{green}F \\
\hline
G & H & I \\
\hline
\end{tabular}
\end{document}
\end{lstlisting}


\begin{tabular}{ | l | l | l | }
	\hline
	A & B & C \\
	\hline
	D & E & \cellcolor{green}F \\
	\hline
	G & H & I \\
	\hline
\end{tabular}
\paragraph{ResizeBox, was hat es damit auf sich?}

Wenn eine Tabelle mal nicht auf die Seite passt oder einfach zu klein ist, kann diese mittels der ResizeBox oder einer ScaleBox an die Seite angepasst werden. In diesem Beispiel wird eine ResizeBox verwendet.

\url{https://en.wikibooks.org/wiki/LaTeX/Tables#Resize_tables}

\begin{lstlisting}[style=Latex,caption={Beispiel für ResizeBox},label=lst:tab10]
\begin{table}[h!]
  \centering
    \resizebox{\textwidth}{!}{
      \begin{tabular}{*{14}{|c}|}%%{|c|c|c|c|c|c|c|c|c|c|c|c|c|c|}
       \hline
        One & Two &Three & Four & Five & Six & Seven & Eight & Nine & Ten & Eleven &
        Twelve & Thirteen & Fourteen\\
       \hline
       \hline
       $1.111$ & $2.222$ & $3.333$ & $4.444$ & $5.555$ & $6.666$ & $7.777$ &
       $8.888$ & $9.999$ & $0.000$ & $1.111$ & $2.222$ & $3.333$ & $4.444$\\
       \hline
       \end{tabular}
    }
\caption{Test Table}
\label{tab:label_test2}
\end{table}
\end{lstlisting}

	\begin{table}[h!]
		\centering
		\resizebox{\textwidth}{!}{
			\begin{tabular}{*{14}{|c}|}%%{|c|c|c|c|c|c|c|c|c|c|c|c|c|c|}
				\hline
				One & Two &Three & Four & Five & Six & Seven & Eight & Nine & Ten & Eleven &
				Twelve & Thirteen & Fourteen\\
				\hline
				\hline
				$1.111$ & $2.222$ & $3.333$ & $4.444$ & $5.555$ & $6.666$ & $7.777$ &
				$8.888$ & $9.999$ & $0.000$ & $1.111$ & $2.222$ & $3.333$ & $4.444$\\
				\hline
			\end{tabular}
		}
		\caption{Test Table}
		\label{tab:label_test}
	\end{table}
\paragraph{Tabelle im Querformat}

Manche Tabellen werden so groß, dass diese nur in einem Querformat dargestellt werden können. Im folgenden Beispiel ist die Tabelle nicht nur gedreht, sondern auch zentriert und skaliert. Es folgt eine leere Seite hinter und vor einer Landscape Tabelle, da das Paket \emph{lscape} automatisch einen Seitenumbruch einfügt. 
\bigskip
\begin{lstlisting}[style=Latex,caption={Einfache gedrehte Tabelle},label=lst:tab11]
\begin{landscape}
  \begin{table}[]
  \centering
   \resizebox{0.5\textwidth}{!}{%
     \begin{tabular}{lllll}
     A	& B  & C  & D  & E  \\
     F	& G & H & I  & J 
     \end{tabular}%
    }
\caption{Eine gedrehte Tabelle. Es ist nicht nötig das Blatt zu drehen.}
\label{tablequer}
\end{table}
\end{landscape}
\end{lstlisting}


\begin{landscape}
	\begin{table}[]
		\centering
		%\resizebox{8cm}{4cm}{%
			\begin{tabular}{l|l|l|l|l}
			A	& B  & C  & D  & E  \\
			F	& G & H & I  & J 
			\end{tabular}%
		%}
		\caption{Eine gedrehte Tabelle. Es ist nicht nötig das Blatt zu drehen.}
		\label{tablelscpe}
	\end{table}
\end{landscape}
\paragraph{Zellen mit anderer Ausrichtung als die gesammte Spalte}
Um eine Zelle mit einer anderen Ausrichtung anzulegen muss diese mit dem Befehl \\ \verb|multicolumn{anzahl}{ausrichtung}{text}| angelegt werden. Näheres darüber auf \url{http://www.weinelt.de/latex/multicolumn.html}
\begin{table}[h]
	\begin{tabular}{|l|r|}
		\hline
		Header1 & HeaderX \\ \hline
		Item1   & X1      \\ 
		Item2   & X2      \\ 
		\multicolumn{1}{|r|}{Item3}   & X3      \\
		\hline
	\end{tabular}
\end{table}