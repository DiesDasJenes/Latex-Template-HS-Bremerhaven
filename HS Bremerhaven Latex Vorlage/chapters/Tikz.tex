\chapter{Diagramme und Graphen in \LaTeX{}}
Durch das Paket Tikz besitzt \LaTeX{} ein Werkzeug um alles zu Zeichnen was möglich ist, sei es Graphen, Diagramme, Bilder, Figuren, 3D-Modelle. Alles ist mit dem Paket möglich. Hier ein paar Beispiele:

\begin{tikzpicture} % Beta Winkel bei einem Dreieck
\draw
(3,-1) coordinate (a) node[right] {A}
-- (0,0) coordinate (b) node[left] {B}
-- (2,2) coordinate (c) node[above right] {C}
pic["$\beta$", draw=orange, <->, angle eccentricity=1.2, angle radius=1cm]
{angle=a--b--c};
\end{tikzpicture}
\hspace{3cm}
\begin{tikzpicture}[thick,scale=1,node distance=1cm] % Graph
\graph[nodes={draw,circle},edges={-{Stealth[]}}] {
	A -> ["1"] B,
	A -> C,
	C -> B
};
\end{tikzpicture}
\bigskip

\tikzset{mynode/.style={inner sep=2pt,fill,outer sep=0,circle}} % Quader mit Fläche
\begin{tikzpicture}[scale=3,line width=1pt]
\coordinate (1) at (0,0,0);
\coordinate (2) at (0,1,0);
\coordinate (3) at (1,1,0);
\coordinate (4) at (1,0,0);
\coordinate (5) at (0,0,1);
\coordinate (6) at (0,1,1);
\coordinate (7) at (1,1,1);
\coordinate (8) at (1,0,1);
\foreach \x in {1,2,...,8}{
	\node[mynode] at (\x) {};
}
\draw (1) -- (2) -- (3) -- (4) -- cycle;
\draw (5) -- (6) -- (7) -- (8) -- cycle;
\draw (1) -- (5)  (2) -- (6) (3) -- (7) (4) -- (8);
\draw[thin,fill=blue!50,opacity=0.5] (0.5,0,0) -- (0.5,0,1) -- (0.5,1,1) -- (0.5,1,0) --cycle;
\node at (0.5,-0.2,1) {(\textit{h\,k\,l})};
\end{tikzpicture}
\hspace{1cm}
\begin{tikzpicture}[scale=3,line width=1pt] % Quader mit zwei Flächen
\coordinate (1) at (0,0,0);
\coordinate (2) at (0,1,0);
\coordinate (3) at (1,1,0);
\coordinate (4) at (1,0,0);
\coordinate (5) at (0,0,1);
\coordinate (6) at (0,1,1);
\coordinate (7) at (1,1,1);
\coordinate (8) at (1,0,1);
\foreach \x in {1,2,...,8}{
	\node[mynode] at (\x) {};
}
\draw (1) -- (2) -- (3) -- (4) -- cycle;
\draw (5) -- (6) -- (7) -- (8) -- cycle;
\draw (1) -- (5)  (2) -- (6) (3) -- (7) (4) -- (8);
\draw[thin,fill=blue!50,opacity=0.5] (1) -- (8) -- (6) --cycle;
\draw[thin,fill=blue!50,opacity=0.5] (2) -- (4) -- (7) --cycle;
\node at (0.5,-0.2,1) {(\textit{h\,k\,l})};
\end{tikzpicture}

\begin{tikzpicture} % Weiterer Graph
\graph {
	A -> { subgraph I_n [V= {B,C,D}] } -> E
};
\end{tikzpicture}
\hspace{1cm}
\begin{tikzpicture}[->,>=stealth',shorten >=2pt, line width=3pt,
node distance=1cm, style ={minimum size=10mm}] % Zustandsübergangsdiagramm einer Markov-Kette
\tikzstyle{every node}=[font=\huge]
\node [circle, draw] (a) {1};
\path (a) edge [loop above] (a);
\node [circle, draw] (b) [right=of a] {2};
\path (b) edge [loop above] (b);
\draw[->] (a) -- (b);
\node [circle, draw] (c) [below=of a] {3};
\path (c) edge [loop below] (c);
\draw[->] (a) -- (c);
\draw[->] (b) -- (c);
\end{tikzpicture}

\newpage
 Diagramm des Lebenszyklus der Android-Aktivität
 \begin{center}
 

\tikzset{%
	>={Latex[width=2mm,length=2mm]},
	% Specifications for style of nodes:
	base/.style = {rectangle, rounded corners, draw=black,
		minimum width=4cm, minimum height=1cm,
		text centered, font=\sffamily},
	activityStarts/.style = {base, fill=blue!30},
	startstop/.style = {base, fill=red!30},
	activityRuns/.style = {base, fill=green!30},
	process/.style = {base, minimum width=2.5cm, fill=orange!15,
		font=\ttfamily},
}
\begin{tikzpicture}[node distance=2cm,
every node/.style={fill=white, font=\sffamily}, align=center]
% Specification of nodes (position, etc.)
\node (start)             [activityStarts]              {Activity starts};
\node (onCreateBlock)     [process, below of=start]          {onCreate()};
\node (onStartBlock)      [process, below of=onCreateBlock]   {onStart()};
\node (onResumeBlock)     [process, below of=onStartBlock]   {onResume()};
\node (activityRuns)      [activityRuns, below of=onResumeBlock]
{Activity is running};
\node (onPauseBlock)      [process, below of=activityRuns, yshift=-1cm]
{onPause()};
\node (onStopBlock)       [process, below of=onPauseBlock, yshift=-1cm]
{onStop()};
\node (onDestroyBlock)    [process, below of=onStopBlock, yshift=-1cm] 
{onDestroy()};
\node (onRestartBlock)    [process, right of=onStartBlock, xshift=4cm]
{onRestart()};
\node (ActivityEnds)      [startstop, left of=activityRuns, xshift=-4cm]
{Process is killed};
\node (ActivityDestroyed) [startstop, below of=onDestroyBlock]
{Activity is shut down};     
% Specification of lines between nodes specified above
% with aditional nodes for description 
\draw[->]             (start) -- (onCreateBlock);
\draw[->]     (onCreateBlock) -- (onStartBlock);
\draw[->]      (onStartBlock) -- (onResumeBlock);
\draw[->]     (onResumeBlock) -- (activityRuns);
\draw[->]      (activityRuns) -- node[text width=6cm]
{Another activity comes in
	front of the activity} (onPauseBlock);
\draw[->]      (onPauseBlock) -- node {The activity is no longer visible}
(onStopBlock);
\draw[->]       (onStopBlock) -- node {The activity is shut down by
	user or system} (onDestroyBlock);
\draw[->]    (onRestartBlock) -- (onStartBlock);
\draw[->]       (onStopBlock) -| node[yshift=1.25cm, text width=3cm]
{The activity comes to the foreground}
(onRestartBlock);
\draw[->]    (onDestroyBlock) -- (ActivityDestroyed);
\draw[->]      (onPauseBlock) -| node(priorityXMemory)
{higher priority $\rightarrow$ more memory}
(ActivityEnds);
\draw           (onStopBlock) -| (priorityXMemory);
\draw[->]     (ActivityEnds)  |- node [yshift=-2cm, text width=3.1cm]
{User navigates back to the activity}
(onCreateBlock);
\draw[->] (onPauseBlock.east) -- ++(2.6,0) -- ++(0,2) -- ++(0,2) --                
node[xshift=1.2cm,yshift=-1.5cm, text width=2.5cm]
{The activity comes to the foreground}(onResumeBlock.east);
\end{tikzpicture}
 \end{center}
\newpage
Weiterführende Anleitungen und Dokumentationen zu Tikz:
\begin{itemize}
	\item \url{http://www.math.uni-leipzig.de/~hellmund/LaTeX/pgf-tut.pdf}
	\item \url{https://cremeronline.com/LaTeX/minimaltikz.pdf}
	\item \url{https://en.wikibooks.org/wiki/LaTeX/PGF/TikZ}
	\item \url{https://www.bu.edu/math/files/2013/08/tikzpgfmanual.pdf}
\end{itemize}

Diese Dokumente zeigen was möglich ist:
\begin{itemize}
	\item\url{http://bit.ly/2pdUfCn}
	\item  \url{http://bit.ly/2G19KHy}
\end{itemize}

Quellen für die Beispiele:
\begin{itemize}
	\item \url{http://www.texample.net/tikz/examples/tag/flowcharts/}
	\item \url{http://bit.ly/2DxhAmL}
	\item \url{https://tex.stackexchange.com/questions/213075/two-tikzpictures-side-by-side}
    \item \url{http://books.goalkicker.com/LaTeXBook/}
\end{itemize}

